\documentclass[a4paper, 12pt]{extarticle}

% paper layout
\usepackage[margin=0.25cm]{geometry}

% multicolumn layout
\usepackage{multicol}
\setlength\columnsep{20pt}
\setlength{\columnseprule}{0.1pt} 

% landscape
\usepackage{pdflscape}

% paragraph layout
\setlength\parindent{0pt}

% language settings
\usepackage[ngerman]{babel} % Silbentrennung
\usepackage[utf8]{inputenc} % Umlaute

\usepackage{titlesec}
%\titleformat*{\section}{\large\bfseries}
%\titleformat*{\subsection}{\large\bfseries}
\titlespacing*{\section}
{5pt}{0ex plus 0ex minus 0ex}{0ex plus 0ex}
\titlespacing*{\subsection}
{5pt}{0ex plus 0ex minus 0ex}{0ex plus 0ex}

% font settings
%\usepackage{mathpazo}
%\usepackage{helvet}
%\usepackage{millennial}
% \usepackage{fouriernc}
%\usepackage[varg]{txfonts}
%\usepackage{mathptmx}
%\usepackage[charter]{mathdesign}
%\usepackage[garamond]{mathdesign}
%\usepackage[utopia]{mathdesign}
%\usepackage{fourier}

% font kerning (load after specific font!)
\usepackage{microtype}

% custom font sizes
\usepackage{relsize}

% underlining with line breaks (umlaut support)
\usepackage{ulem}

% graphics settings
\usepackage{pict2e} % load this before picture
\usepackage{picture}
\usepackage{graphicx}
\usepackage{caption}
\usepackage{subcaption}

% graphics drawings
\usepackage{tikz}
\usetikzlibrary{calc,matrix,trees}

% Set the overall layout of the tree
\tikzstyle{level 1}=[level distance=2cm, sibling distance=2cm]
\tikzstyle{level 2}=[level distance=2cm, sibling distance=1cm]

% Define styles for bags and leafs
\tikzstyle{bag} = [text width=4em, text centered]
\tikzstyle{end} = [circle, minimum width=3pt,fill, inner sep=0pt]

% float settings
\usepackage{float}

% referencing
\usepackage{hyperref}

% math packages and settings
\usepackage[intlimits]{mathtools}
\usepackage{amssymb}
\usepackage{resizegather} % resizing equations
\usepackage{oubraces}    % special overlapping over/underbraces
\usepackage{cancel}
\allowdisplaybreaks % allow page break in align* environment

% tables
\usepackage{booktabs} % nicer tables
\usepackage{array} % custom column types

% lists
\usepackage{enumitem}
\setlist{noitemsep,topsep=3pt,parsep=3pt,partopsep=3pt,leftmargin=15pt}

% list items
\renewcommand\labelitemi{{\boldmath$\cdot$}}
\newcommand{\listarrow}{
\smash{\scalebox{1.5}[1.75]{\rotatebox[origin=c]{180}{$\Lsh$}}}
}

% comments (invisible)
\usepackage{comment}

% comments (visible comment)
\newcommand{\txtcom}[1]{\textcolor{gray}{\{ #1 \}}}

% separator
\newcommand{\sep}{\vspace{5pt}\noindent\hrule\vspace{5pt}}

% color packages
\usepackage{color}
\usepackage{xcolor}

% ASYMPTOTIC NOTATIONS
\newcommand{\BigO}{\mathcal{O}}

% NUMBER SYSTEMS
\newcommand{\E}{\mathbb{E}}
\newcommand{\K}{\mathbb{K}}
\newcommand{\F}{\mathbb{F}}
\newcommand{\N}{\mathbb{N}}
\newcommand{\Z}{\mathbb{Z}}
\newcommand{\Q}{\mathbb{Q}}
\newcommand{\R}{\mathbb{R}}
\newcommand{\C}{\mathbb{C}}

% CALLICGRAPHIC SHORTCUTS
\newcommand{\cA}{\mathcal{A}}
\newcommand{\cB}{\mathcal{B}}
\newcommand{\cC}{\mathcal{C}}
\newcommand{\cD}{\mathcal{D}}
\newcommand{\cE}{\mathcal{E}}
\newcommand{\cF}{\mathcal{F}}
\newcommand{\cG}{\mathcal{G}}
\newcommand{\cH}{\mathcal{H}}
\newcommand{\cI}{\mathcal{I}}
\newcommand{\cJ}{\mathcal{J}}
\newcommand{\cK}{\mathcal{K}}
\newcommand{\cL}{\mathcal{L}}
\newcommand{\cM}{\mathcal{M}}
\newcommand{\cN}{\mathcal{N}}
\newcommand{\cO}{\mathcal{O}}
\newcommand{\cP}{\mathcal{P}}
\newcommand{\cQ}{\mathcal{Q}}
\newcommand{\cR}{\mathcal{R}}
\newcommand{\cS}{\mathcal{S}}
\newcommand{\cT}{\mathcal{T}}
\newcommand{\cU}{\mathcal{U}}
\newcommand{\cV}{\mathcal{V}}
\newcommand{\cW}{\mathcal{W}}
\newcommand{\cX}{\mathcal{X}}
\newcommand{\cY}{\mathcal{Y}}
\newcommand{\cZ}{\mathcal{Z}}

% SETS
\newcommand{\set}[1]{\left\{ #1 \right\}}
\newcommand{\dset}[2]{\left\{ #1 \ \middle| \ #2 \right\}}

% BRACES
\newcommand{\alg}[1]{\left\langle #1 \right\rangle}
\newcommand{\card}[1]{\left\lvert #1 \right\rvert}
\newcommand{\length}[1]{\left\lvert #1 \right\rvert}
\newcommand{\abs}[1]{\left\lvert #1 \right\rvert}
\newcommand{\norm}[1]{\left\lVert #1 \right\rVert}
\newcommand{\scprod}[1]{\left\langle #1 \right\rangle}
\newcommand{\ceil}[1]{\left\lceil #1 \right\rceil}
\newcommand{\floor}[1]{\left\lfloor #1 \right\rfloor}
\newcommand{\linsys}[2]{\left[\ #1 \ | \ #2 \ \right]}

% BRACES SMALL
\newcommand{\snorm}[1]{\lVert #1 \rVert}
\newcommand{\sscprod}[1]{\langle #1 \rangle}
\newcommand{\slinsys}[2]{[\ #1 \ | \ #2 \ ]}

% DISJOINT UNION
\makeatletter
\def\moverlay{\mathpalette\mov@rlay}
\def\mov@rlay#1#2{\leavevmode\vtop{%
   \baselineskip\z@skip \lineskiplimit-\maxdimen
   \ialign{\hfil$\m@th#1##$\hfil\cr#2\crcr}}}
\newcommand{\charfusion}[3][\mathord]{
    #1{\ifx#1\mathop\vphantom{#2}\fi
        \mathpalette\mov@rlay{#2\cr#3}
      }
    \ifx#1\mathop\expandafter\displaylimits\fi}
\makeatother
\newcommand{\bigcupdot}{\charfusion[\mathop]{\bigcup}{\cdot}}
\newcommand{\cupdot}{\mathbin{\mathaccent\cdot\cup}}

% CUSTOM STATISTICS
\newcommand{\Prob}[2][]{\operatorname{P}_{#1}\left[ #2 \right]}
\newcommand{\cProb}[2]{\operatorname{P}\left[ #1 \,\middle|\, #2 \right]}
\newcommand{\Var}[2][]{\operatorname{Var}_{#1}\left[ #2 \right]}
\newcommand{\sd}[1]{\operatorname{sd}\left( #1 \right)}
\newcommand{\Exp}[2][]{\operatorname{E}_{#1}\left[ #2 \right]}
\newcommand{\Corr}[1]{\operatorname{Corr}\left[ #1 \right]}
\newcommand{\Cov}[1]{\operatorname{Cov}\left(#1 \right)}
\newcommand{\MSE}[2][]{\operatorname{MSE}_{#1}\left[ #2 \right]}
\newcommand{\riid}{\stackrel{\text{\tiny i.i.d.}}{\sim}}
\newcommand{\approxsim}{\stackrel{\text{\tiny approx.}}{\sim}}

% ACCENTS
\newcommand*{\Hm}{\mathsf{H}}
\newcommand*{\T}{\mathsf{T}}
\newcommand*{\Rev}{\mathsf{R}}

% CUSTOM ALPHABETS
\renewcommand{\S}{\Sigma}
\newcommand{\Ss}{\Sigma^*}
\newcommand{\Sp}{\Sigma^+}
\newcommand{\Sbool}{\Sigma_{\text{bool}}}
\newcommand{\Ssbool}{(\Sigma_{\text{bool}})^*}
\newcommand{\Slogic}{\Sigma_{\text{logic}}}
\newcommand{\Sslogic}{(\Sigma_{\text{logic}})^*}
\newcommand{\Slat}{\Sigma_{\text{lat}}}
\newcommand{\Sslat}{(\Sigma_{\text{lat}})^*}
\newcommand{\Stastatur}{\Sigma_{\text{Tastatur}}}
\newcommand{\Sstastatur}{(\Sigma_{\text{Tastatur}})^*}
\newcommand{\Sm}{\Sigma_{m}}
\newcommand{\Ssm}{\Sigma_{m}^*}
\newcommand{\ZO}{\{0,1\}}
\newcommand{\ZOs}{\{0,1\}^*}
\newcommand{\hdelta}{\hat\delta}

% OPERATORS
\DeclareMathOperator{\id}{\text{id}}
\DeclareMathOperator{\Kon}{\text{Kon}}
\DeclareMathOperator{\cost}{\text{cost}}
\DeclareMathOperator{\goal}{\text{goal}}
\DeclareMathOperator{\Opt}{\text{Opt}}
\DeclareMathOperator{\Bin}{\text{Bin}}
\DeclareMathOperator{\Nummer}{\text{Nummer}}
\DeclareMathOperator{\Prim}{\text{Prim}}
\DeclareMathOperator{\Kl}{\text{Kl}}
\DeclareMathOperator{\lcm}{lcm}
\DeclareMathOperator{\glb}{glb}
\DeclareMathOperator{\lub}{lub}
\DeclareMathOperator{\im}{im}
\DeclareMathOperator{\ord}{ord}
\DeclareMathOperator{\rank}{rank}
\DeclareMathOperator{\spn}{span}
\DeclareMathOperator{\sign}{sign}
\DeclareMathOperator{\Trace}{Trace}
\DeclareMathOperator{\Spur}{Spur}
\DeclareMathOperator{\diag}{diag}
\DeclareMathOperator{\arcsinh}{arcsinh}
\DeclareMathOperator{\arccosh}{arccosh}
\DeclareMathOperator{\arctanh}{arctanh}
% \DeclareMathOperator{\arc}{arc} % TODO: is now a circle of some other package
\renewcommand\div{\operatorname{div}}
\DeclareMathOperator{\rot}{rot}
\DeclareMathOperator{\cis}{cis}
\DeclareMathOperator{\grad}{grad}
\DeclareMathOperator{\Hess}{Hess}
\newcommand{\laplace}{\Delta}

% OPERATORS (OVERRIDDEN)
\renewcommand\Re{\operatorname{Re}}
\renewcommand\Im{\operatorname{Im}}

% RELATIONS
\newcommand{\mbeq}{\stackrel{!}{=}}
\newcommand{\relid}{\mathrel{\id}}
\newcommand{\relrho}{\mathrel{\rho}}
\newcommand{\relsigma}{\mathrel{\sigma}}
\newcommand{\reltheta}{\mathrel{\theta}}
\newcommand{\relsim}{\mathrel{\sim}}
\newcommand{\relf}{\mathrel{f}}

% RELATIONS (INVERSES)
\newcommand{\invrelid}{\mathrel{\widehat{\id}}}
\newcommand{\invrelrho}{\mathrel{\widehat{\rho}}}
\newcommand{\invrelsigma}{\mathrel{\widehat{\sigma}}}
\newcommand{\invreltheta}{\mathrel{\widehat{\theta}}}
\newcommand{\invrelsim}{\mathrel{\widehat{\sim}}}
\newcommand{\invrelf}{\mathrel{\widehat{f}}}

% CUSTOM RELATIONS
\DeclareRobustCommand{\step}[2][]{\mathrel{\drawstep{#1}{#2}}}
\newcommand{\drawstep}[2]{%
  \vcenter{\hbox{%
    \setlength{\unitlength}{1em}%
    \begin{picture}(1,1)
    \roundcap
    \put(0,0){\line(0,1){1}}
    \put(0,0.5){\line(1,0){0.95}}
    \put(0.5,0){\makebox[0pt]{\text{\smaller$\scriptscriptstyle#2$}}}
    \put(0.5,0.6){\makebox[0pt]{\text{\smaller$#1$}}}
    \end{picture}%
  }}%
}

% GLOBAL MATRICES AND VECTOR SETTINGS
\newcommand{\mat}[1]{\mathbf{#1}}
%\renewcommand{\vec}[1]{\mathbf{#1}}

% VECTORS
\newcommand{\va}{\vec{a}}
\newcommand{\vb}{\vec{b}}
\newcommand{\vc}{\vec{c}}
\newcommand{\vd}{\vec{d}}
\newcommand{\ve}{\vec{e}}
\newcommand{\vf}{\vec{f}}
\newcommand{\vg}{\vec{g}}
\newcommand{\vh}{\vec{h}}
\newcommand{\vi}{\vec{i}}
\newcommand{\vj}{\vec{j}}
\newcommand{\vk}{\vec{k}}
\newcommand{\vl}{\vec{l}}
\newcommand{\vm}{\vec{m}}
\newcommand{\vn}{\vec{n}}
\newcommand{\vo}{\vec{o}}
\newcommand{\vp}{\vec{p}}
\newcommand{\vq}{\vec{q}}
\newcommand{\vr}{\vec{r}}
\newcommand{\vs}{\vec{s}}
\newcommand{\vt}{\vec{t}}
\newcommand{\vu}{\vec{u}}
\newcommand{\vv}{\vec{v}}
\newcommand{\vw}{\vec{w}}
\newcommand{\vx}{\vec{x}}
\newcommand{\vy}{\vec{y}}
\newcommand{\vz}{\vec{z}}

% MATRICES (LATIN)
\newcommand{\MA}{\mat{A}}
\newcommand{\MB}{\mat{B}}
\newcommand{\MC}{\mat{C}}
\newcommand{\MD}{\mat{D}}
\newcommand{\ME}{\mat{E}}
\newcommand{\MF}{\mat{F}}
\newcommand{\MG}{\mat{G}}
\newcommand{\MH}{\mat{H}}
\newcommand{\MI}{\mat{I}}
\newcommand{\MJ}{\mat{J}}
\newcommand{\MK}{\mat{K}}
\newcommand{\ML}{\mat{L}}
\newcommand{\MM}{\mat{M}}
\newcommand{\MN}{\mat{N}}
\newcommand{\MO}{\mat{0}}
\newcommand{\MP}{\mat{P}}
\newcommand{\MQ}{\mat{Q}}
\newcommand{\MR}{\mat{R}}
\newcommand{\MS}{\mat{S}}
\newcommand{\MT}{\mat{T}}
\newcommand{\MU}{\mat{U}}
\newcommand{\MV}{\mat{V}}
\newcommand{\MW}{\mat{W}}
\newcommand{\MX}{\mat{X}}
\newcommand{\MY}{\mat{Y}}
\newcommand{\MZ}{\mat{Z}}

% MATRICES (GREEK)
\newcommand{\MSigma}{\mat{\Sigma}}
\newcommand{\MLambda}{\mat{\Lambda}}

% HIGHLIGHTING IN EQUATIONS
\newcommand{\hl}[1]{\colorbox{black!7}{$#1$}}

% programming codes
% TODO: use better programming package: pygments
\usepackage{listings}
\usepackage{courier}
\definecolor{light-gray}{gray}{0.9}
\lstset{
 %rulecolor=\color{black},
 %frame=single, % adds a frame around the code
 basicstyle        = \footnotesize\ttfamily,
 framextopmargin   = 50pt
 breaklines        = true,
 breakatwhitespace = true,
 %backgroundcolor=\color{light-gray},
 breakindent       = 2ex,
 escapechar       = *,   % custom escape char
 %numbers          = left
 tabsize           = 2,
 %literate={\ \ }{{\ }}1
 mathescape        = true   % escape math with '$'
}

% todos
\newcommand{\todo}[1]{\textcolor{red}{\textbf{TODO:} #1}}

% TODO: warmup environment (short/long version of document)
% TODO: algorithm package (abstract)
% TODO: graph packages (tikz)
% TODO: minted codes (and language-specific)
% TODO: gnuplot, pgfplots examples
% TODO: displayskip, parskip (what is this?)
% TODO: ultra shortening option for document

% CUSTOM ARGUMENTS
\newcommand{\argdot}{\,\cdot\,}

% custom commands

% package for conditional actions
\usepackage{xifthen}

% color definitions
\definecolor{defcolor}{rgb}{0.75, 1, 0.75}
\definecolor{thmcolor}{rgb}{1, 0.75, 0.75}
\definecolor{excolor}{rgb}{1, 0.95, 0.43}
%\definecolor{importantcolor}{rgb}{0.7, 0.95, 1} % light blue
\definecolor{importantcolor}{rgb}{1, 0.55, 0}
\definecolor{titlecolor}{rgb}{0, 0, 0}

% conditional checks
\newcommand{\emptyarg}[1][]{\ifthenelse{\isempty{#1}}{}{\ (#1)}}

% custom commands with optional argument
\newcommand{\Def}[1][]{\colorbox{defcolor}{\color{titlecolor}{\textbf{Def.\emptyarg[#1]}}}\kern+0.3ex}
\newcommand{\Thm}[1][]{\colorbox{thmcolor}{\color{titlecolor}{\textbf{Thm.\emptyarg[#1]}}}\kern+0.3ex}
\newcommand{\Lem}[1][]{\colorbox{thmcolor}{\color{titlecolor}{\textbf{Lem.\emptyarg[#1]}}}\kern+0.3ex}
\newcommand{\Cor}[1][]{\colorbox{thmcolor}{\color{titlecolor}{\textbf{Cor.\emptyarg[#1]}}}\kern+0.3ex}
\newcommand{\Ex}[1][]{\colorbox{excolor}{\color{titlecolor}{\textbf{Ex.\emptyarg[#1]}}}\kern+0.3ex}

% custom commands with no argument
\newcommand{\Com}{\textbf{Com.} }
\newcommand{\Important}{\textbf{Wichtig:} }
\newcommand{\Attention}{\textbf{Achtung:} }
\newcommand{\Proof}{\textbf{Bew.} }
\newcommand{\Intuition}{\textbf{Intuition:} }
\newcommand{\Trick}{\textbf{Trick:} }

% custom proof commands
\newcommand{\qed}{\hfill$\square$}
\newcommand{\contradiction}{\hfill$\lightning$}

% codes
\lstnewenvironment{Code}[1][]{}{}        % boxed
\newcommand{\code}[1]{\lstinline{#1}}    % inline



% title
\title{Wahrscheinlichkeit und Statistik}
\author{Zusammenfassung}
\date{}

\begin{document}


% define hidden column type for tables
\newcolumntype{H}{>{\setbox0=\hbox\bgroup}c<{\egroup}@{}}


\begin{multicols*}{2}
\raggedcolumns

\section{Allgemeine Zufallsvariablen}

\sep

\Def[Verteilungsfunktion]
\[
t\mapsto F_X(t):=\Prob{X\leq t}:=\Prob{\dset{\omega\in\Omega}{X(\omega)\leq t}}.
\]

\sep

Jede Verteilungsfunktion $F_X$ hat die folgenden \emph{Eigenschaften}:
\begin{enumerate}[label=\arabic*)]
  \item $F_X$ ist i) \emph{wachsend}, d.h., es gilt $F_X(s)\leq F_X(t)$ für
  $s\leq t$, und ii) \emph{rechtsstetig}, d.h., es gilt $F_X(u)\to F_X(t)$ für
    $u\to t$ mit $u>t$.
  \item $\displaystyle \lim_{t\to - \infty}F_X(t)=0$, und
  $\displaystyle \lim_{t\to + \infty}F_X(t)=1$.
\end{enumerate}
Umgekehrt kann man zeigen, dass jede Funktion $F$ mit den Eigenschaften 1) und
2) die Verteilungsfunktion $F_X$ einer Zufallsvariablen $X$ ist.

\sep

Kennt man die Verteilung kennt man die Verteilungsfunkt. und umgekehrt
$\mu_X(B):=\Prob{X\in B}$ und $F_X(t)=\mu_X\left(\left(-\infty,
t\right]\right)$.

\sep

\Def[gemeinsame Verteilungsfunktion] 

ZV $X_1,\ldots,X_n$, Abb. $F\colon\R^n\to[0,1]$,
\begin{gather*}
(x_1,\ldots,x_n)\mapsto F(x_1,\ldots,x_n):=\Prob{X_1\leq x_1, \ldots, X_n\leq
x_n}.
\end{gather*}
\[
F(x_1,\ldots,x_n)=\int_{-\infty}^{x_1}\cdots\int_{-\infty}^{x_n}
f(t_1,\ldots,t_n) \ dt_n \ldots dt_1
\]

\sep

Falls $X_1,\ldots,X_n$ eine gemeinsame Dichte $f$ haben, so gilt analog zum
diskreten Fall
\begin{enumerate}[label=\roman*)]
  \item $f(x_1,\ldots,x_n)\geq 0$, und $=0$ ausserh. v.
  $\cW(\ldots)$.
  \item $\int_{-\infty}^\infty\cdots\int_{-\infty}^{\infty} f(x_1,\ldots,x_n) \
  dx_n \ldots dx_1=1.$
  \item $\Prob{(X_1,\ldots,X_n)\in
  A}=\substack{\int\cdots\int}_{(x_1,\ldots,x_n)\in A}f(x_1,\ldots,x_n) \
  dx_n\ldots dx_1$ für $A\subseteq \R^n$.
\end{enumerate}

\sep

\Def[Verteilungsfkt. der Randverteilung von $X$] 

ZV $X$, $Y$, $F_X\colon\R\to[0,1]$
\begin{gather*}
x\mapsto F_X(x):=\Prob{X\leq x}=\Prob{X\leq x, Y<\infty}=\lim_{y\to\infty}
F(x,y)
\end{gather*}
\Com $F_Y\colon\R\to[0,1]$ analog.

\sep

\Def[Dichtefunktion einer Randverteilung] 

ZV $X,Y$ mit gemeinsamer Dichte $f(x,y)$\\
Randverteilung von $X$ ist $f_X\colon\R\to[0,\infty)$
\[
f_X(x)=\int_{-\infty}^\infty f(x,y) \ dy	
\]
\Com $f_Y(y)$ analog.

\sep

\Def[Unabhängigkeit] Die ZV $X_1,\ldots,X_n$ heissen
\emph{unabhängig}, falls gilt
\[
F(x_1,\ldots,x_n)=F_{X_1}(x_1)\cdots F_{X_n}(x_n), \text{ bzw.}
\]
\[
f(x_1,\ldots,x_n) = f_{X_1}(x_1)\cdots f_{X_n}(x_n)\quad \text{für alle
}x_1,\ldots,x_n,
\]

\sep

\Def[Faltung von stetigen Zufallsvariablen] 

Sind $X$ und $Y$ zusätzlich unabhängig, so ist $f(x,y)=f_X(x)f_Y(y)$ und damit
\[
f_Z(z)=\int_{-\infty}^{\infty} f_X(x)f_Y(z-x) \ dx
=\int_{-\infty}^{\infty} f_X(z-y)f_Y(y) \ dy
\]
d.h. $f_Z=(f_X * f_Y)(z)$ ist die \emph{Faltung} $f_X$
und $f_Y$.

\sep

\Def[Bedingte Gewichtsfkt. und Vert.] 

ZV $X,Y$, gegeben
\begin{gather*}
p_{X|Y}(x\,|\,y):=\cProb{X=x}{Y=y}
=\frac{\Prob{X=x,Y=y}}{\Prob{Y=y}}
=\frac{p(x,y)}{p_Y(y)}
\end{gather*}
für $p_Y(y)>0$ und $0$ sonst. Das legt dann auch die bedingte Verteilung von $X$
gegeben $Y=y$ fest.

\sep

\Lem[Affine Transformation von ZV]

Sei $g(x)=ax+b$ mit $a>0$, $b\in\R$ und $Y=g(X) = aX+b$. Es folgt
\begin{align*}
F_Y(t)&=\Prob{Y\leq t} = \Prob{aX+b\leq t}
\\ 
&= \Prob{X\leq \frac{t-b}{a}}=F_X\left(\frac{t-b}{a}\right)
\end{align*}
und damit nach der Kettenregel
\[
f_Y(t) = \frac{d}{dt}F_Y(t) = \frac{1}{a} f_X\left(\frac{t-b}{a}\right).
\]
Analog bis auf ein Vorzeichen geht das auch mit $a<0$.

\sep

\Ex[Affine Transform. von normalvert. ZV]

Ist $X\sim\cN(\mu,\sigma^2)$ und $Y=aX+b$, $a>0$, $b\in\R$, so ist
$Y\sim\cN(b+a\mu,a^2\sigma^2)$. (Bew: obige Formel.)

\sep

\Ex[Nichtlineare Transformation 1] 

Für $g(x)=x^2$ ist $Y=X^2$ und
\begin{gather*}
\begin{align*}
F_Y(t) &= \Prob{Y\leq t} = \Prob{X^2\leq t}
=\Prob{-\sqrt{t}\leq X\leq \sqrt{t}}
\\
&= \Prob{X\leq\sqrt{t}} - \Prob{X\leq-\sqrt{t}}
=F_X(\sqrt{t}) - F_X(-\sqrt{t}),
\end{align*}
\end{gather*}
falls $X$ stetig ist. Mit d. Kettenregel erhält man $f_Y(t)$.

\sep

\Ex[Nichtlineare Transformation 2] 

Für $g(x)=\frac{1}{x}$ ist $Y=\frac{1}{X}$ und
\begin{align*}
F_Y(t) &= \Prob{Y\leq t} = \Prob{\frac{1}{X}\leq t} \stackrel{(*)}{=}
\Prob{X\geq\frac{1}{t}}
\\
&= 1 - \Prob{X\leq \frac{1}{t}} = 1-F_X\left(\frac{1}{t}\right),
\end{align*}
falls $X$ stetig ist. $f_Y(t)$ $\to$ Ableiten und Kettenregel.

(*): Keine Mult. Einfach Bruch umkehren.

\sep

\Thm[Vert. mit Umkehrfkt. und $X\sim\cU(0,1)$] 

Sei $F$ eine stetige und streng monoton wachsende Verteilungsfunktion, mit
Umkehrfunktion $F^{-1}$. Ist $X\sim\cU(0,1)$ und $Y=F^{-1}(X)$, so hat $Y$
gerade die Verteilungsfunktion $F$.

\sep

\Ex[Simulation der Exponentialverteilung]

Um eine Exponentialverteilung mit Parameter $\lambda$ zu simulieren, nehmen wir
die zugehörige Verteilungsfunktion $F(t)=1-e^{-\lambda t}$ für $t\geq 0$. Ihre
Inverse $F^{-1}$ erhalten wir via
\[
y=F(t)=1-e^{-\lambda t}= 1-e^{-\lambda F^{-1}(y)}
\]
als
\[
F^{-1}(y)=t=-\frac{\ln(1-y)}{\lambda}.
\]
Mit $X\sim\cU(0,1)$ ist also
\[
Y:=F^{-1}(X) = -\frac{\ln(1-X)}{\lambda} \sim Exp(\lambda).
\]

\sep

\section{Ungleichungen und\\ Grenzwertsätze}

\sep

\subsection{Ungleichungen}

\sep

\Thm[Markov Ungleichung] Sei $X$ eine Zufallsvariable und ferner
$g\colon\cW(X)\to[0,\infty)$ eine wachsende Funktion. Für jedes $c\in\R$ mit
$g(c)>0$ gilt dann
\[
\Prob{X\geq c} \leq \frac{\Exp{g(X)}}{g(c)}.
\]

\sep

\Cor[Chebyshev-Ungleichung] Sei $Y$ eine Zufallsvariable mit endlicher Varianz.
Für jedes $b>0$ gilt dann
\[
\Prob{\abs{Y-\Exp{Y}}\geq b} \leq \frac{\Var{Y}}{b^2},
\]
bzw.
\[
\Prob{\abs{Y-\Exp{Y}}\leq b} \geq 1- \frac{\Var{Y}}{b^2}.
\]

\sep

Die obigen Ungleichungen illustrieren die Bedeutung der Varianz als
Streunungsmass: Je kleiner die Varianz, desto eher (mit desto grösserer
Wahrscheinlichkeit) liegen die Werte von $X$ nahe beim Erwartungswert $\Exp{X}$.

\sep

\subsection{Das Gesetz der grossen Zahlen\\ \small(Asymptotik von
Zufallsvariablen)}

\sep

\Thm[schwaches Gesetz der grossen Zahlen] Sei $X_1,X_2,\ldots$ eine Folge von
unabhängigen (es genügt schon: paarweise unkorreliert) Zufallsvariablen, die
alle den gleichen Erwartungswert $\Exp{X_i}=\mu$ und die gleiche Varianz
$\Var{X_i}=\sigma^2$ haben. Sei
$\overline{X}_n=\frac{1}{n}S_n=\frac{1}{n}\sum_{i=1}^n X_i$. Dann konvergiert
$\overline{X}_n$ für $n\to\infty$ in Wahrscheinlichkeit/stochastisch gegen
$\mu=\Exp{X_i}$, d.h.
\[
\Prob{\abs{\overline{X}_i-\mu}>\epsilon}
\xrightarrow{n \to \infty} 0
 \qquad \text{für jedes }\epsilon >0.
\]
Anschaulich bedeutet die obige Aussage, dass mit beliebig grosser
Wahrscheinlichkeit $1-\delta$ der Wert von $\overline{X}_n$ für hinreichend
grosse $n$ beliebige nahe (bis auf $\epsilon$) bei $\mu$ liegt.

\sep

\Thm[starkes Gesetz der grossen Zahlen] Sei $X_1,X_2,\ldots$ eine Folge von
unabhängigen Zufallsvariablen, die alle dieselbe Verteilung haben, und ihr
Erwartungswert $\mu = \Exp{X_i}$ sei endlich. Für
$\overline{X}_n=\frac{1}{n}S_n=\frac{1}{n}\sum_{i=1}^n X_i$ gilt dann
\[
\overline{X}_n \xrightarrow{n \to \infty} \mu
\quad
P\text{-fastsicher }(P\text{-f.s.}),
\]
d.h.
\[
\Prob{\dset{\omega\in\Omega}{\overline{X}_n(\omega)
\xrightarrow{n \to \infty}\mu}}=1.
\]

\sep

\subsection{Der Zentrale Grenzwertsatz\\ \small(Asymptotik von Verteilungen)}

\sep

\Thm[Zentraler Grenzwertsatz, ZGS] 

Sei $X_1,X_2,\ldots$ eine Folge von i.i.d. Zufallsvariablen mit $\Exp{X_i}=\mu$
und $\Var{X_i}=\sigma^2$. Für die Summe $S_n=\sum_{i=1}^n X_i$ gilt dann
\[
\lim_{n\to\infty}\Prob{\frac{S_n-n\mu}{\sigma\sqrt{n}}\leq x}=\Phi(x)
\quad
\text{für alle }x\in\R,
\]
wobei $\Phi$ die Verteilungsfunktion der $\cN(0,1)$-Verteilung ist.

\sep

\Com Wie im Beweis des schwachen GGZ kann man nachrechnen, dass $S_n$
Erwartungswert $\Exp{S_n}=n\mu$ und Varianz $\Var{S_n}=n\sigma^2$ hat. Also hat
die Grösse
\[
S^*_n = \frac{S_n-n\mu}{\sigma\sqrt{n}} = \frac{S_n-\Exp{X}}{\sqrt{\Var{S_n}}}
\]
Erwartungswert $0$ und Varianz $1$; sie heisst deshalb auch die
\emph{Standardisierung} von $S_n$.

\sep

\Ex[Normalapprox. von Summen von ZV:] Für
\emph{praktische Anwendungen} schreibt man die Aussage des ZGS meistens in der Form
\[
\Prob{S^*_n\leq x} \approx \Phi(x) 
\ \ \ \text{oder} \ \ \
S^*_n\approxsim \cN(0,1)
\ \ \ \text{für }n\text{ gross}
\]
wobei das Symbol $\approxsim$ für ``ist approximativ
verteilt gemäss'' steht. Für $S_n$ oder $\overline{X}_n=\frac{1}{n}S_n$
zurückübersetzt heisst das dann
\[
S_n \approxsim \cN(n\mu,n\sigma^2)
\quad\text{bzw.}\quad
\overline{X}_n \approxsim \cN(\mu,\frac{1}{n}\sigma^2),
\]
und das wird benutzt, um Wahrscheinlichkeiten für diese Grössen (d.h. für $S_n$
bzw. $\overline{X}_n$) approximativ als Wahrscheinlichkeiten für eine
entsprechende Normalverteilung zu berechnen.

\sep

\Ex[Normalapprox. für die Binomialvert.] Ist $n$ gross, so wird
es schwierig die Binomialkoeffizienten bei der Dichte/Verteilungsfunktion der
Binomialverteilung zu berechnen. Deshalb fasst man die Binomialverteilung
(sofern es möglich ist) als Summe von i.i.d. $X_1,\ldots,X_n\sin Be(p)$
Zufallsvariablen auf, d.h.
\begin{gather*}
S_{n}=\sum_{i=1}^{n} X_i \sim Bin(n,p),
\quad
\Exp{X_i}=p,
\quad
\Var{X_i}=p(1-p).
\end{gather*}
Die Approximation ergibt sich dann mithilfe des ZGS für grosse $n$
folgendermassen:
\[
S_{n}\approxsim\cN\left(n \Exp{X_i},
n\Var{X_i}\right)=\cN\left(np,np(1-p)\right)
\]
und natürlich ist die standardisierte Zufallsvariable folgendermassen verteilt:
\[
S_{n}^*=\frac{S_n-np}{\sqrt{np(1-p)}}=\frac{S_n - \Exp{S_n}}{\sqrt{\Var{S_n}}}
\approxsim \cN(0,1).
\]
Eine etwas genauere Approximation für die Binomialverteilung erhält man, wenn
man zusätzlich noch die sogenannte \emph{Kontinuitätskorrektur} benutzt:
\begin{gather*}
\begin{align*}
\Prob{a<S_n\leq b} &=
\Prob{\frac{a-np}{\sqrt{np(1-p)}}<S_n^*\leq\frac{b-np}{\sqrt{np(1-p)}}}
\\
&\approx
\Phi\left(\frac{b+\frac{1}{2}-np}{\sqrt{np(1-p)}}\right)
-
\Phi\left(\frac{a+\frac{1}{2}-np}{\sqrt{np(1-p)}}\right)
\end{align*}
\end{gather*}
Korrektur $+\frac{1}{2}$: Zentrieren der Stäbe im Histogramm macht Approximation
genauer.

\sep

\subsection{Grosse Abweichungen und\\ Chernoff-Schranken}

\sep

\Def[Momenterzeugende Funktion] einer Zufallsvariable $X$ ist
\[
M_X(t)=\Exp{e^{tX}} \quad \text{für }t\in\R;
\]
das ist immer wohldefiniert in $[0,\infty]$, kann aber $+\infty$ werden.

\sep

\Thm Seien $X_1,\ldots,X_n$ i.i.d. Zufallsvariablen, für welche die
momenterzeugende Funktion $M_X(t)$ für alle $t\in\R$ endlich ist. Für jedes
$b\in\R$ gilt dann
\[
\Prob{S_n\geq b} \leq \exp\left(\inf_{t\in\R} \left(n\log M_X(t)
-tb\right)\right).
\]

\sep

\Thm Seien $X_1,\ldots,X_n$ unabhängig mit $X_i\sim Be(p_i)$ und
$S_n=\sum_{i=1}^n X_i$. Sei ferner $\mu_n:=\Exp{S_n}=\sum_{i=1}^n p_i$ und
$\delta > 0$. Dann gilt
\[
\Prob{S_n\geq (1+\delta)\mu_n} \leq
\left(\frac{e^\delta}{(1+\delta)^{(1+\delta)}}\right)^{\mu_n}.
\]

\Com Sind $X_1,\ldots,X_n\riid Be(p)$, so ist $S_n\sim Bin(n,p)$ und man kann es
für die Abschätzung von Restwahrscheinlichkeiten von der Binomialverteilung
anwenden.

\sep

\section{Statistische Grundideen}

\sep

\Def[Stichprobe, Stichprobenumfang] Die Gesamtheit der Beobachtungen
$x_1,\ldots,x_n$ oder Zufallsvariablen $X_1,\ldots,X_n$ nennt man oft eine
\emph{Stichprobe}; die Anzahl $n$ heisst dann der \emph{Stichprobenumfang}.

\sep

\section{Schätzer}

\sep

\Def[Erwartungstreu] Ein Schätzer $T$ heisst \emph{erwartungstreu} für
$\vartheta$, falls gilt $\Exp[\vartheta]{T}=\vartheta$; im Mittel (über alle
denkbaren Realisationen $\omega$) schätzt $T$ also richtig. Allgemein heisst
$\Exp[\vartheta]{T}-\vartheta$ der \emph{Bias} (oder erwartete Schätzfehler) von
$T$; erwartungstreu (auf Englisch ``\emph{unbiased}'') bedeutet also, dass der
Bias Null ist.

\sep

\Def[Mean Squared Error, MSE] Der \emph{mittlere quadratische Schätzfehler}
(``\emph{mean squared error}'', MSE) ist definiert als
\[
\MSE[\vartheta]{T}:=\Exp[\vartheta]{(T-\vartheta)^2}.
\]

\sep

\Def[Konsistent] Eine Folge von Schätzern $T^{(n)}$, $n\in\N$ heisst
\emph{konsistent} für $\vartheta$, falls $T^{(n)}$ für $n\to\infty$ in
$P_\vartheta$-Wahrscheinlichkeit gegen $\vartheta$ konvergiert, d.h. für jedes
$\vartheta\in\Theta$ gilt
\[
\lim_{n\to\infty} \Prob[\vartheta]{\abs{T^{(n)}-\vartheta}>\epsilon} = 0
\qquad
\text{für jedes }\epsilon>0.
\]
(Das setzt anschaulich voraus, dass man beliebig viele Daten haben könnte.)

\sep

\Com Mithilfe der Rechenregeln für Erwartungswerte kann man den MSE zerlegen in
\begin{gather*}
\begin{align*}
\MSE[\vartheta]{T}&=\Exp[\vartheta]{(T-\vartheta)^2}
\\
&=\Exp[\vartheta]{T^2} -2\vartheta\Exp[\vartheta]{T} + \vartheta^2
\\
&=\Exp[\vartheta]{T^2} - (\Exp[\vartheta]{T})^2 + (\Exp[\vartheta]{T})^2 
-2\vartheta\Exp[\vartheta]{T} + \vartheta^2
\\
&=\Var[\vartheta]{T} + (\Exp[\vartheta]{T}-\vartheta)^2,
\end{align*}
\end{gather*}
also in die \emph{Summe aus der Varianz des Schätzers $T$ und dem Quadrat des
Bias}. \emph{Für erwartungstreue Schätzer sind Varianz und MSE also dasselbe
(weil der zweite Summand wegfällt)}.

\sep

\subsection{Maximum-Likelyhood-Methode}
\sep

\Def[Likelyhood-Funktion]
\begin{gather*}
L(x_1,\ldots,x_n;\vartheta):=\begin{cases}
p(x_1,\ldots,x_n;\vartheta)&\text{im diskreten
Fall},\\
f(x_1,\ldots,x_n;\vartheta)&\text{im stetigen Fall}.
\end{cases}
\end{gather*}
$\log L(x_1,\ldots,x_n;\vartheta)$ heisst
\emph{log-Likelihood-Funktion}.

\sep

\Def[Maximum-Likelyhood-Schätzer] Der ML-Schätzer $T$ für $\vartheta$ wird 
dadurch definiert, dass er
\[
\vartheta \mapsto L(x_1,\ldots,x_n;\vartheta) \qquad
\text{als Funktion von }\vartheta
\]
maximiert.

\sep

Meistens sind $X_1,\ldots,X_n$ i.i.d. unter $P_\vartheta$; die
Likelihood-Funktion $L$ ist dann ein Produkt, und es ist bequemer, statt $L$ die
log-Likelihood-Funktion $\log L$ zu maximieren, weil diese eine Summe ist. Statt
zu maximieren sucht man ferner meistens nur Nullstellen der (partiellen)
Ableitung(en) nach den Komponenten von $\vartheta$.

\sep

\Def[Momentenschätzer] Der Schätzer $T=(T_1,T_2)$, wobei
$
T_1=\frac{1}{n}\sum_{i=1}^n X_i = \overline{X}_n$,
$
T_2=\frac{1}{n}\sum_{i=1}^n (X_i-\overline{X}_n)^2
=\left(\frac{1}{n}\sum_{i=1}^n X_i^2\right) - \left(\overline{X}_n\right)^2 
$
ist der sogenannte \emph{Momentenschätzer} für
$
(\Exp[\vartheta]{X},\Var[\vartheta]{X})
$
in jedem Modell $P_\vartheta$, wo $X_1,\ldots,X_n$ i.i.d. sind. 

\sep

\Cor[Momentenschätzer nicht erwartungstreu] 

Der Momentenschätzer hat aber den allgemeinen Nachteil, dass er nicht
erwartungstreu für $(\Exp[\vartheta]{X},\Var[\vartheta]{X})$ ist. Zwar ist für
jedes $\vartheta$
$
\Exp[\vartheta]{T_1}=\Exp[\vartheta]{X}$; aber
$\Exp[\vartheta]{T_2}=\frac{n-1}{n}\Exp[\vartheta]{X}$.
Um einen erwartungstreuen Schätzer $T'$ für
$(\Exp[\vartheta]{X},\Var[\vartheta]{X})$ zu haben benutzt man deshalb meistens
$T_1'=T_1=\overline{X}_n$,\\
$T_2'=\frac{n}{n-1}T_2 
= \frac{1}{n-1}\sum_{i=1}^n(X_i-\overline{X}_n)^2
=\frac{1}{n-1}\left(\sum_{i=1}^n X_i^2\right)
-\frac{n}{n-1}\left(\overline{X}_n\right)^2$.

\sep

\Def[Empirische Stichprobenvarianz] Für $T_2'$ benutzt man oft auch die Notation
\[
S^2:=\frac{1}{n-1}\sum_{i=1}^n(X_i-\overline{X}_n)^2
\]
und nennt $S^2$ die \emph{empirische Stichprobenvarianz}.

\sep

\subsection{Verteilungsaussagen}

\sep

In vielen Situationen ist es nützlich oder nötig die Verteilung (unter
$P_\vartheta$, für jedes $\vartheta\in\Omega$) eines Schätzers zu kennen. Exakte
allgemeine Aussagen gibt es dazu nur wenige. Einen allgemeine
\emph{approximativen Zugang} liefert der Zentrale Grenzwertsatz (siehe vorher).
Für normalverteilte Stichproben hat man aber folgenden Satz:

\sep

\Thm Seien $X_1,\ldots,X_n\riid \cN(\mu,\sigma^2)$. Dann gilt:
\begin{enumerate}[label=\arabic*)]
  \item $\overline{X}_n$ ist normalverteilt gemäss
  $\sim\cN(\mu,\frac{1}{n}\sigma^2)$, und
  $\frac{\overline{X}_n-\mu}{\sigma/\sqrt{n}}\sim\cN(0,1)$.
  \textcolor{red}{ (Richtige Var. im Nenner nehmen!)}
  \item $\frac{n-1}{\sigma^2}S^2 = \frac{1}{\sigma^2}\sum_{i=1}^n
  (X_i-\overline{X}_n)^2$ is $\chi^2$-verteilt mit $n-1$ Freiheitsgraden.
  \item $\overline{X}_n$ und $S^2$ sind unabhängig.
  \item Der Quotient
  \[
  \frac{\overline{X}_n-\mu}{S/\sqrt{n}} =
  \frac{\frac{\overline{X}_n-\mu}{\sigma/\sqrt{n}}}{S/\sigma}
  =\frac{\frac{\overline{X}-\mu}{\sigma/\sqrt{n}}}{
  \sqrt{\frac{1}{n-1}\frac{n-1}{\sigma^2}S^2}}
  \]
  ist $t$-verteilt mit $n-1$ Freiheitsgraden.
\end{enumerate}

\sep

\section{Tests}

\sep

\textbf{Grundidee:} Wir haben schon eine Vermutung, wo in $\Theta$ der richtige
(aber unbekannte) Parameter $\vartheta$ liegen könnte, und wollen diese mit
Hilfe der Daten überprüfen (``testen'').

\sep

\Def[Test, Teststatistik] Die Zufallsvariable
$T=t(X_1,\ldots,X_n)$ heisst \emph{Teststatistik}. Die Entscheidungsregel
bzw. der \emph{Test} ist $I_{\set{t(x_1,\ldots,x_n)\in K}}$. D.h. verwerfe $H_0$
falls $I(\omega)=1$.

\sep

Grundsätzlich versucht man Tests immer so zu konstruieren, dass die folgenden
Wahrscheinlichkeiten möglichst klein werden. Eine Möglichkeit dafür ist das
folgende \textbf{asymmetrische Vorgehen}: Wähle ein Signifikanzniveau $\alpha$
und versuche in zwei Schritten die folgenden Fehlerwahrscheinlichkeiten möglichst
klein zu machen

\begin{enumerate}[label=\alph*)]
  \item \textbf{Fehler 1. Art:} $H_0$ wird zu unrecht abgelehnt (nicht
  akzeptiert). Um diese Fehlerwhs. zu minimieren muss man
  schauen, dass die folgende Whs. möglichst klein (kleiner als das
  vorgegebene $\alpha$) wird:
  \[
  \sup \Prob[\vartheta\in\Theta_0]{T\in K} < \alpha 
  \quad \text{(Signifikanzniveau)}
  \]
  \item \textbf{Fehler 2. Art:} $H_0$ wird zu unrecht akzeptiert (nicht
  verworfen). Um diese Fehlerwhs. zu minimieren muss man schen, dass  die
  folgende Whs möglichst klein, bzw. die Macht des Tests $\beta(\vartheta)$
  möglichst gross wird:
  \[
  \inf \Prob[\vartheta\in\Theta_A]{T\not\in K} = 1 - \sup
  \underbrace{\Prob[\vartheta\in\Theta_A]{T\in K}}_{=\beta(\vartheta) \quad
  (\vartheta\in\Theta_A)}
  \]
\end{enumerate}

Dieses asymmetrische Vorgehen impliziert folgendes:
\begin{itemize}
  \item Es ist schwieriger $H_0$ zu verwerfen. Deshalb wählt man in der Praxis
  für $H_0$ die Negation der Aussage (gelingt es trotzdem $H_0$ zu verwerfen
  kann man zuversichtlicher sein).
  \item Es kann durchaus sein, dass man durchs Vertauschen von $H_0$ und $H_A$
  unterschiedliche Testergebnisse erhält.
\end{itemize}

\Com Es ist nicht immer möglich die Verteilung von $T$ unter jedem
$\vartheta\in\Theta_A$ für die Macht des Tests zu bestimmen. Dann gibt man sich
damit zufrieden, dass man nur die Verteilung von $T$ unter
$\vartheta\in\Theta_A$ zum Einhalten des Signifikanzniveaus bestimmt, um das
Signifikanzniveau einzuhalten.

\sep

\Def[P-Wert] Der Wert wird bestimmt durch die gezogene Stichprobe. Er deutet an,
wie wahrscheinlich es ist, ein solches Stichprobenergebnis oder ein extremeres
zu erhalten, wenn die Nullhypothese wahr ist. Ein häufiges Missverständnis ist
die Gleichsetzung dieser Aussage mit der falschen Behauptung, der p-Wert würde
angeben, wie wahrscheinlich die Nullhypothese bei Erhalt dieses
Stichprobenergebnisses ist. Mit dem p-Wert wird also angedeutet, wie extrem das
Ergebnis ist: je kleiner der p-Wert, desto mehr spricht das Ergebnis gegen die
Nullhypothese.

\resizebox{1\linewidth}{!}{
\begin{tabular}{lccc}
&Links & Beidseitig & Rechts\\
$p$-Wert=
&
$\Prob[\vartheta_0]{X\leq x}$ 
&
$\begin{cases}
\Prob[\vartheta_0]{x\geq X}, & T(\omega)>\mu_0\\
\Prob[\vartheta_0]{X\geq x}, & T(\omega)<\mu_0\\
\end{cases}$
& 
$\Prob[\vartheta_0]{x\leq X}$
\end{tabular}
}
Falls wir ein Signifikanzlevel $\alpha$ a priori gegeben haben, dann gilt also
\begin{itemize}
  \item verwerfe $H_0$, falls $p$-Wert $\leq\alpha$, und
  \item akzeptiere $H_0$, falls $p$-Wert $>\alpha$.
\end{itemize}

\sep

\subsection{Konstruktion von Tests}

\sep

\Def[Likelihood Quotient] Sei $L(x_1,\ldots,x_n;\vartheta)$ die
Likelihood-Funktion; für diskrete $X_i$ ist das also im Modell $P_\vartheta$ die
Wahrscheinlichkeit $\Prob[\vartheta]{X_1=x_1,\ldots,X_n=x_n}$, die beobachteten
Werte $x_1,\ldots,x_n$ zu erhalten. Für $\vartheta_0\in\Theta_0$ und
$\vartheta_A\in\Theta_A$ betrachten wir den \emph{Likelihood-Quotienten} 
\[
R(x_1,\ldots,x_n;\vartheta_0,\vartheta_A):=
\frac{L(x_1,\ldots,x_n;\vartheta_0)}{L(x_1,\ldots,x_n;\vartheta_A)}.
\]
\emph{Ist dieser Quotient klein, so ist der Nenner wesentlich grösser als der
Zähler.
Das bedeutet also, dass die Beobachtungen $x_1,\ldots,x_n$ als Resultate im
Modell $P_{\vartheta_A}$ deutlich wahrscheinlicher sind als im Modell
$P_{\vartheta_0}$; die Daten sprechen also gegen $\vartheta_0$ im Vergleich zu
$\vartheta_A$}.

\emph{Es liegt deshalb nahe, als Teststatistik
$T:=R(X_1,\ldots,X_n;\vartheta_0,\vartheta_A)$ und als kritischen Bereich
$K:=[0,c)$ zu wählen, wenn man $\vartheta_0$ gegen $\vartheta_A$ testen will;
man verwirft dann also die Hypothese $H_0$, wenn der Quotient $R$ zu klein wird.
Wegen $R\geq 0$ genügt $[0,c)$ statt $(-\infty,c)$.}

\sep

Der letzte Schritt den wir allgemein noch machen müssen ist, um den Kritischen
Bereich $K$ passend zum gewünschten Niveau $\alpha$ festlegen zu können,
brauchen wir die Verteilung der Teststatistik $T$ unter der Hypothese $H_0$,
d.h. in jedem Modell $P_\vartheta$ mit $\vartheta\in\Theta_0$. Evtl. 

\sep

Sind Hypothese und Alternative beide einfach (was leider eher selten der Fall
ist), so ist dieser Test optimal, wie das folgende Resultat zeigt.

\sep

\Thm[Neyman Pearson-Lemma] Sei $\Theta_0=\set{\vartheta_0}$ und
$\Theta_A=\set{\vartheta_A}$. Wie oben sei
$T:=R(x_1,\ldots,x_n;\vartheta_0,\vartheta_A)$ und $K:=[0,c)$, sowie
$\alpha^*=\Prob[\vartheta_0]{T\in K}=\Prob[\vartheta_0]{T\leq c}$. Der
\emph{Likelihood-Quotienten-Test} mit Teststatistik $T$ und kritischem Bereich
$K$ ist dann im folgenden Sinn optimal: Jeder andere Test mit Signifikanzniveau
$\alpha\leq\alpha^*$ hat kleinere Macht bzw. eine grössere Wahrscheinlichkeit
für einen Fehler 2. Art. 

(Etwas formaler: Ist $(T',K')$ ein anderer Test mit $\Prob[\vartheta_0]{T'\in
K'}\leq \alpha^*$, so gilt auch $\Prob[\vartheta_A]{T'\in K'}\leq
\Prob[\vartheta_A]{T\in K}$.)

\sep

\Def[Verallgemeinerter Likelihood-Quotient] 
\[
R(x_1,\ldots,x_n):=\frac{
\sup_{\vartheta \in\Theta_0}L(x_1,\ldots,x_n;\vartheta)
}{
\sup_{\vartheta \in\Theta_A}L(x_1,\ldots,x_n;\vartheta)
}
\]
oder auch
\[
\tilde{R}(x_1,\ldots,x_n):=\frac{
\sup_{\vartheta \in\Theta_0}L(x_1,\ldots,x_n;\vartheta)
}{
\sup_{\vartheta \in\Theta_A\cupdot\Theta_0}L(x_1,\ldots,x_n;\vartheta)
}
\]
Die Teststatistik ist dann $T_0:=R(X_1,\ldots,X_n)$ bzw.
$\tilde{T}:=\tilde{R}(X_1,\ldots,X_n)$ mit krit. Ber. $K_0:=[0,c_0)$.

\sep

\section{Konfidenzbereiche}

\sep

\textbf{Grundidee:} Wie in vorherigen Abschnitt suchen wir aus einer Familie
$(P_\vartheta)_{\vartheta\in\Theta}$ von Modellen eines, das zu unseren Daten
$x_1,\ldots,x_n$ passt. Ein Schätzer für $\vartheta$ gibt uns dabe einen
einzelnen zufälligen möglichen Parameterwert. Weil es schwierig ist, mit diesem
einen Wert den richtigen (aber unbekannten) Paremeter zu treffen, suchen wir nun
stattdessen eine \emph{(zufällige) Teilmenge des Parameterbereichs}, die
hoffentlich den wahren Parameter enthält.

\sep

\Def[Konfidenzbereich] Etwas formaler ist ein \emph{Konfidenzbereich für
$\vartheta$} zu Daten $x_1,\ldots,x_n$ eine Menge
$C(x_1,\ldots,x_n)\subseteq\Theta$; in den meisten Fällen ist das ein Intervall,
dessen Endpunkte von $x_1,\ldots,x_n$ abhängen. Ersetzen wir die Daten
$x_1,\ldots,x_n$ durch die sie generierenden Zufallsvariablen, so ist
$C(X_1,\ldots,X_n)$ also eine zufällige Teilmenge von $\Theta$, mit Realisierung
$C(\omega)=C(X_1(\omega),\ldots,X_n(\omega))$ bei einem festen $\omega$. Ein
solches $C$ heisst ein \emph{Konfidenzbereich zum Niveau $1-\alpha$}, falls gilt
\[
\Prob[\vartheta]{\vartheta\in C(X_1,\ldots,X_n)}\geq 1-\alpha
\qquad
\text{für alle }\vartheta\in\Theta
\]
d.h. in jedem Modell erwischt man den Parameter mit grosser Wahrscheinlichkeit.

\sep

Seien die Daten die Realisationen von Zufallsvariablen $X_1,\ldots,X_n$, die
alle unter $P_\vartheta$ i.i.d. $\sim\cN(\mu,\sigma^2)$ sind. Die
offensichtlichen \emph{Schätzer} für $(\mu,\sigma^2)$ sind
$T=(\overline{X}_n,S^2)$. Das Konfidenzintervall wird um diese Schätzer herum
angesetzt.

\begin{enumerate}[label=\alph*)]
  \item \emph{$\mu$ gesucht, $\sigma^2$ bekannt}: Wir wissen, dass in diesem
  Fall $\frac{\overline{X}_n-\mu}{\sigma/\sqrt{n}}\sim \cN(0,1)$ unter
  $P_\vartheta$. Also ist das Konfidenzintervall für $\mu$ zum Niveau $1-\alpha$
  \begin{gather*}
  C(X_1,\ldots,X_n)=\left[
  \overline{X}_n-z_{1-\frac{\alpha}{2}}\frac{\sigma}{\sqrt{n}},
  \overline{X}_n+z_{1-\frac{\alpha}{2}}\frac{\sigma}{\sqrt{n}}
  \right].
  \end{gather*}
  \item \emph{$\mu$ gesucht, $\sigma^2$ gesucht}: Wir wissen, dass in diesem
  Fall $\frac{\overline{X}_n-\mu}{S/\sqrt{n}}\sim t_{n-1}$ unter $P_\vartheta$.
  Also ist das Konfidenzintervall für $\mu$ zum Niveau $1-\alpha$
  \begin{gather*}
  C(X_1,\ldots,X_n)=\left[
  \overline{X}_n-t_{n-1,1-\frac{\alpha}{2}}\frac{S}{\sqrt{n}},
  \overline{X}_n+t_{n-1,1-\frac{\alpha}{2}}\frac{S}{\sqrt{n}}
  \right].
  \end{gather*}
  Weil $\frac{1}{\sigma^2}(n-1)S^2=\frac{1}{\sigma^2}\sum_{i=1}^n
  (X_i-\overline{X}_n)^2\sim\chi_{n-1}^2$ unter $P_\vartheta$ ist, ist das
  Konfidenzinterval für $\sigma^2$ zum Nieveau $1-\alpha$
  \begin{gather*}
  C(X_1,\ldots,X_n)=\left[
  \frac{(n-1)S^2}{\chi^2_{n-1,1-\frac{\alpha}{2}}},
  \frac{(n-1)S^2}{\chi^2_{n-1,\frac{\alpha}{2}}}
  \right].
  \end{gather*}
  \item \emph{$\mu$ bekannt, $\sigma^2$ gesucht}: Wir bestimmen das
  Konfidenzinterval für $\sigma^2$ zum Niveau $1-\alpha$ genau wie bei b).
\end{enumerate}

\sep

\Important Man bemerke, dass das Konfidenzintervall in allen Fällen schmäler
wird, sobald der Stichprobenumfang $n$ grösser wird!

\sep

\Com Können wir keine genauen Verteilungsaussagen über die Zufallsvariablen
$X_1,\ldots,X_n$ machen, so ist es schwieriger exakte Konfidenzintervalle zu
erhalten. In allgemeinen Situationen kann man oft nur approximative
Konfidenzintervalle mit Hilfe des Zentralen Grenzwertsatzes bekommen (siehe
vorher).
\sep

\textbf{Vom Konfidenzbereich zur Hypothese:}
\[
\begin{matrix}
C(X_1,\ldots,X_n)\text{ ist}\\
\text{Konfidenzbereich für}\\
\vartheta \text{ zum Niveau } 1-\alpha
\end{matrix}
\Longrightarrow
\begin{matrix}
I_{\set{\vartheta_o\not\in C(X_1,\ldots,X_n)}} \text{ ist}\\
\text{Test für }\vartheta=\vartheta_0\\
\text{zum Niveau }\alpha
\end{matrix}
\]

\sep

\textbf{Von der Hypothese zum Konfidenzbereich:}

\[
\begin{matrix}
I_{\set{t(x_1,\ldots,x_n) \in K_{\vartheta_0}}}\\
\text{Test für }\vartheta=\vartheta_0\\
\text{zum Niveau }\alpha
\end{matrix}
\Longrightarrow
\begin{matrix}
C(X_1,\ldots,X_n):=K_\vartheta ^c\\
\text{ist Konfidenzbereich für}\\
\vartheta \text{ zum Niveau } 1-\alpha
\end{matrix}
\]

\sep

\section{Sonstiges}

\sep

\textbf{Drawing Elements from a Set:} The table shows the umber of possibilities
for selecting $k$ elements from a set of size $n$ (the example set is
$\{a,b,c\}$ thus $n=3$; and $k=2$):

\begin{table}[H]
\resizebox{\columnwidth}{!}{
\begin{tabular}{|l|cc|cc|}
\toprule
 & \multicolumn{2}{c|}{\textbf{Ordered}} &
 \multicolumn{2}{c}{\textbf{Unordered}}\\
 & Number & Examples & Number & Examples\\
\midrule
\shortstack{w.\\ rep.} & $\begin{matrix}n^k\\\text{(Variat.)}\end{matrix}$ &
$\begin{matrix}
aa & ab & ac\\
ba & bb & bc\\
ca & cb & cc
\end{matrix}$
& 
$\binom{n+k-1}{k}$
&
$\begin{matrix}
aa & ab & ac\\
bb & bc &   \\
cc\\
\end{matrix}$\\
\hline
\shortstack{w.o.\\ rep.} &
$n^{\underline{k}}$
&
$\begin{matrix}
ab & ac \\
ba & bc \\
ca & cb\\
\end{matrix}$
&
$\begin{matrix}\binom{n}{k}\\\text{\small(Kombin.)}\end{matrix}$
&
$\begin{matrix}
ab & ac & bc\\
\end{matrix}$\\
\bottomrule
\end{tabular}
}
\end{table}

\sep

\textbf{Identitäten} $\binom{n}{k}=\binom{n}{n-k}$, \quad
$\binom{n}{k}=\binom{n-1}{k-1}+\binom{n-1}{k}$,
\\
$(x+y)^n=\sum_{k=0}^n\binom{n}{k}x^{n-k}y^k$, \quad
$\sum_{k=0}^n\binom{n}{k}=2^n$,\\
$\sum_{k=0}^n(-1)^k\binom{n}{k}=0$, 

\sep

\textbf{Geometrische Reihe}
\[
S_n = a_0\sum_{k=0}^n q^k = a_0\frac{1-q^{n+1}}{1-q} = a_0 \frac{q^{n+1}-1}{q-1}
\]
\[
S = a_0\sum_{k=0}^\infty = \frac{1}{1-q} \qquad \text{ falls } \abs{q}<1
\]
\Important Auf die Indizes achten! Sie beginnt bei $k=0$ somit ist das erste Glied
1. Falls bei $k=1$ begonnen wird muss man $-1$ addieren!

\sep

\textbf{Arithmetische Reihe}
\[
S_n = \sum_{k=0}^{n}\left(k\cdot d + a_0\right)= (a_0+a_n) \cdot \frac{(n+1)}{2}
\]
wobei
\[
a_i = i\cdot d + a_0 \quad \forall i\geq 1
\]
und wobei $d$ die Differenz zwischen zwei Folgegliedern ist.

\sep

\Ex[Summe der Zahlen von 0 bis $n$]
\[
\sum_{k=0}^n k = n\cdot \frac{(n+1)}{2}
\]

\sep

\Ex[Summe der ersten $n$ ungeraden Zahlen]
\begin{gather*}
\sum_{k=1}^n(2k-1) = \sum_{k=0}^{n-1}(2k+1) = 1 + 3 + 5 + 7 +\cdots+ (2n-1) =
n^2
\end{gather*}

\sep

\textbf{Summe von Quadratzahlen}

Die Summe der ersten $n$ Quadratzahlen ist für alle $n\geq 1$:
\[
\sum_{k=1}^n k^2 = \frac{n \cdot (n+1) \cdot (2n+1)}{6}.
\]

\sep

\textbf{Summe von spezieller Potenzreihe}

Die Summe für die folgende Potenzreihe für ein $q$ 
mit $\abs{q}<1$ ist:
\begin{align*}
\sum_{k=1}^\infty k\cdot q^{k-1}
&=
\sum_{k=1}^\infty \frac{d}{d q} (q^k)
=
\frac{d}{d q}\left(\sum_{k=1}^\infty q^k\right)
\\
&=
\frac{d}{d q}\left(\frac{1}{1-q} -1\right)
=
\frac{1}{(1-q)^2}.
\end{align*}

\Com Oft substituiert man $q:=(1-p)$.


\sep

\textbf{Reihenentwicklung von $e^x$}

$e^x = \sum_{n=0}^\infty \frac{x^n}{n!} = 1 + x + \frac{x^2}{2!}
 + \frac{x^3}{3!} + \cdots,$

\sep

\textbf{Partielle Integration}

$
\int_a^b f'(x)g(x) \ dx = \left[f(x)g(x)\right]_a^b - \int_a^b f(x)g'(x) \ dx
$
\sep

\textbf{Substitution}

Ist $f$ stetig und $g$ ein Diffeomorphismus (in beiden Richtungen stetig
differenzierbare Abbildung), dann gilt
$
\int_a^b f(g(x))g'(x) \ dx = \int_{g(a)}^{g(b)} f(y) \ dy
$

$
y = g(x)	\Longleftrightarrow x = g^{-1}(y)
$

$
\frac{dy}{dx} = g'(x) \Longleftrightarrow dy = g'(x) dx
$

\sep

$
a + b = \max(a,b) + \min(a,b)
$

\sep

Beim Berechnen der Varianz verwendet man oft den Trick dass
$\Var{X}=\Exp{X^2}-(\Exp{X})^2 = \Exp{X(X-1)}+\Exp{X}-(\Exp{X})^2$, weil der
Erwartungswert $\Exp{X(X-1)}$ in manchen Fällen einfacher als $\Exp{X^2}$ zu
berechnen ist.

\sep

\end{multicols*}

\newlength{\beschrXdiskrL}
\setlength{\beschrXdiskrL}{9cm}

\begin{landscape}
\resizebox{1\linewidth}{!}{
\begin{tabular}{cm{\beschrXdiskrL}cccccHH}
\\\addlinespace[0.25em]\toprule\addlinespace[0.25em]
\begin{tabular}{c}
\textbf{Diskrete}\\
\textbf{Verteilung $\mu_X$}
\end{tabular} 
&
\multicolumn{1}{c}{\textbf{Beschrieb von $X$}} &
\textbf{$\cW(X)=$} & 
\begin{tabular}{c}
\textbf{Gewichtsfunktion}\\
\textbf{$p_X(k)=\Prob{X=k}=$}
\end{tabular} &
\begin{tabular}{c}
\textbf{Verteilungsfunktion}\\
\textbf{$F_X(t)=\Prob{X\leq t}=$}
\end{tabular} &
\textbf{$\Exp{X}=$} & 
\textbf{$\Var{X}=$} &
\textbf{Plots (einzeln, cum)} &
\textbf{Beispiele}
\\\addlinespace[0.25em]\midrule\addlinespace[0.25em]
\begin{tabular}{c}
Gleichverteilung\\
$X\sim\cU_T$
\end{tabular} &
\begin{tabular}{p{\beschrXdiskrL}}
Alle $n$ Elementarereignisse der Trägermenge $T$ sind gleich wahrscheinlich.
\end{tabular}
&
$\set{x_1,x_2,\ldots,x_n}$&
$\begin{cases}
\frac{1}{n} & x\in\cW(X),\\
0,& x\not\in\cW(X).
\end{cases}$&
$\displaystyle \frac{\card{\dset{x_k\in\cW(X)}{x_k\leq t}}}{n}$ 
&
$\displaystyle \frac{1}{n}\sum_{i=1}^n x_i$
&
$\frac{1}{n}\left(\sum_{i=1}^n x_i^2 -(x_n)^2\right)$
&
&
%Ergebnis beim Würfeln, Münzwurf, zufällige dreistellige Zahl, zufällige
%Permutation von $\set{1,\ldots,n}$ ($N=n!$)
\\\addlinespace[0.25em]\hline\addlinespace[0.25em]
\begin{tabular}{c}
Bernoulli\\
$\displaystyle X\sim Be(p)$
\end{tabular} 
&
\begin{tabular}{p{\beschrXdiskrL}}
Anzahl der Erfolge $X$ bei einem einzelnen 0-1-Experiment $A_1$ mit
Erfolgsparameter $p$
\\\addlinespace[0.25em]
\multicolumn{1}{c}{$X=Y_1=I_{A_1}$\qquad (Bem.: $X^2=X$)}
\end{tabular}
&
\begin{tabular}{c}
$\set{0,1}$
\\
\multicolumn{1}{l}{1=Erfolg,}\\
\multicolumn{1}{l}{0=Misserfolg.}
\end{tabular}
&
$\displaystyle p^k(1-p)^{1-k}$
&
$ \begin{cases}
0 & t < 0 \\
1-p & 0 \leq t < 1 \\
1 & t \geq 1
\end{cases}
$ 
&
$p$ 
&
$p(1-p)$ 
&
&
\\\addlinespace[0.25em]\hline\addlinespace[0.25em]
\begin{tabular}{c}
Binomial\\
$\displaystyle X\sim Bin(n,p)$
\\
\\
Rekursionsformel:\\
$\Prob{X=k+1}=$\\
$\frac{p}{1-p}\frac{n-k}{k+1}\Prob{X=k}$
\end{tabular} 
& 
\begin{tabular}{p{\beschrXdiskrL}}
Anzahl der Erfolge $X$ bei $n$ unabhängigen 0-1-Experimenten mit
Erfolgsparameter $p$. Sind $Y_1,\ldots,Y_n\riid Be(p)$, so ist
\\\addlinespace[0.25em]
\multicolumn{1}{c}{
$\displaystyle X=\sum_{i=1}^{n} I_{A_i} =\sum_{i=1}^{n}Y_i\sim Bin(n,p).$
}
\\\addlinespace[0.25em]
Approx.: $X\approxsim\cP(np)$ gut falls $np^2\leq 0.05$.
\end{tabular}
&
$\displaystyle \set{0,1,\ldots,n}$ 
&
$\displaystyle \binom{n}{k}p^k(1-p)^{n-k}$
&
$\displaystyle \sum_{k=0}^{\floor{t}} \binom{n}{k}p^k(1-p)^{n-k}$ 
&
$np$ 
&
$np(1-p)$ 
&
&
\\\addlinespace[0.25em]\hline\addlinespace[0.25em]
\begin{tabular}{c}
Geometrisch\\
$\displaystyle X\sim Geom(p)$ \\
\\
``gedächtnislos''
\end{tabular} 
&
\begin{tabular}{p{\beschrXdiskrL}}
Wartezeit $X$ auf den ersten Erfolg bei einer unendlichen Folge von
0-1-Experimenten mit Erfolgsparameter $p$. Spezialfall von $NB(1,p)$ 
\\\addlinespace[0.25em]
\multicolumn{1}{c}{
$\displaystyle
\ X=\inf\dset{i\in\N}{A_i\text{ tritt ein}}
$}
\\\addlinespace[0.25em]
\multicolumn{1}{c}{
$
=\inf\dset{n\in\N}{Y_i=1}$
}
\end{tabular} 
&
$\set{1,2,3,\ldots}$ 
&
$(1-p)^{k-1}p$ 
&
\begin{tabular}{c}
$1-(1-p)^{\floor{t}}$
\\\addlinespace[0.75em]
$\Prob{X>t} = (1-p)^{\floor{t}}$
\end{tabular}
&
$\displaystyle \frac{1}{p}$ 
&
$\displaystyle \frac{1-p}{p^2}$ 
&
\\\addlinespace[0.25em]\hline\addlinespace[0.25em]
\begin{tabular}{c}
Negativbinomial\\
$\displaystyle X\sim NB(r,p)$
\end{tabular} 
&
\begin{tabular}{p{\beschrXdiskrL}}
Wartezeit $X$ auf den $r$-ten Erfolg bei einer unendlichen Folge von
0-1-Experimenten mit Erfolgsparameter $p$. \\
\multicolumn{1}{c}{
$X=\inf\dset{k\in\N}{\sum_{i=1}^k I_{A_i}=r}$
}
\\\addlinespace[0.25em]
\multicolumn{1}{c}{
$\ \ =\inf\dset{k\in\N}{\sum_{i=1}^k Y_i=r}$
}
\\\addlinespace[0.25em]
Sind die ZV $X_1,\ldots,X_r\riid Geom(p)$, so ist $X:=X_1+\cdots+X_r\sim
NB(r,p)$.
\end{tabular} 
& 
$\set{r,r+1,r+2,\ldots}$
&
$\displaystyle\binom{k-1}{r-1} p^r(1-p)^{k-r}$ 
&
$\displaystyle \sum_{k=r}^{\floor{t}} \binom{k-1}{r-1} p^r(1-p)^{k-r}$ 
&
$\displaystyle \frac{r}{p}$ &
$\displaystyle \frac{r(1-p)}{p^2}$ &
\\\addlinespace[0.25em]\hline\addlinespace[0.25em]
\begin{tabular}{c}
Hypergeometrisch\\
$\displaystyle X\sim HypG(n,r,m)$\\
\\
($r$ wie ``richtig")\\
\end{tabular} 
&
\begin{tabular}{p{\beschrXdiskrL}}
In einer Urne seien $n$ Gegenstände, davon $r$ vom Typ 1 und $n-r$ vom Typ 2.
Man zieht ohne Zurücklegen $m$ der Gegenstände; die Zufallsvariable $X$
beschreibt die Anzahl der Gegenstände vom Typ 1 in dieser Stichprobe vom Umfang
$m$.
\\
$T=T_1\cupdot T_2$, $\card{T}=n$, $\card{T_1}=r$, $\card{T_2}=n-r$
\\\addlinespace[0.25em]
$S\subseteq T$, $\card{S}=m$
\\\addlinespace[0.25em]
\multicolumn{1}{c}{
$X=\card{\dset{t\in S}{t\in T_1}}$
}
\end{tabular} 
&
$\set{0,1,\ldots,\min(m,r)}$ 
&
$\displaystyle \frac{\displaystyle
\binom{r}{k}\binom{n-r}{m-k}}{\displaystyle \binom{n}{m}} $ 
&
$\displaystyle \sum_{k=0}^{\floor{t}}
\frac{\displaystyle\binom{r}{k}\binom{n-r}{m-k}}{\binom{n}{m}}$ 
& 
$\displaystyle\frac{mr}{n}$ 
& 
$m \frac{r}{n} \left( 1-\frac{r}{n} \right) \frac{n-m}{n-1}$ 
&
&
Im Schweizer Lotto, wo man 6 aus 45 Zahlen richtig tippen soll, ist die Anzahl
der richtig getippten Zahlen bei einem einzelnen Tipp hypergeometrisch verteilt
mit Parametern $n=45$, $r=6$, und $m=6$.
\\\addlinespace[0.25em]\hline\addlinespace[0.25em]
\begin{tabular}{c}
Poisson\\
$\displaystyle X\sim \cP(\lambda)$
\end{tabular} &
Anzahl Ereignisse $X$ in einem festen Zeitintervall. Der Param. $\lambda>0$ ist
dabei die durchschnittl., bzw. die erwartete Anz. Ereignisse in dem
Intervall. 
& $\set{0,1,2,\ldots}$ 
&
$\displaystyle e^{-\lambda}\frac{\lambda^k}{k!}$ 
&
$\displaystyle e^{-\lambda}\sum_{k=0}^{\floor{t}}\frac{\lambda^k}{k!}$ 
&
$\lambda$ 
& 
$\lambda$ 
&
Anzahl Anrufe bei einer Telefonzentrale in einer bestimmten Periode; Anzahl der
Grossschäden bei einer Versicherung in einer bestimmten Periode; die Anzahl der
Jobs, die bei einem Server in einer bestimmten Periode eintreffen.
\\
\bottomrule
\end{tabular}
}


\newpage

\newlength{\beschrXstetL}
\setlength{\beschrXstetL}{4.8cm}

\resizebox{1\linewidth}{!}{
\begin{tabular}{cm{\beschrXstetL}cccccHH}
\toprule\addlinespace[0.25em]
\begin{tabular}{c}
\textbf{Stetige}\\
\textbf{Verteilung $\mu_X$}
\end{tabular} 
& 
\textbf{Beschrieb von $X$} 
&
\textbf{$\cW(X)=$} 
&
\begin{tabular}{c} 
\textbf{Dichtefunktion}
\\
$f_X(t)=$\ $\color{red} (\neq\Prob{X=t})$
\end{tabular}
&
\begin{tabular}{c}
\textbf{Verteilungsfunktion}\\
\textbf{$F_X(t)=\Prob{X\leq t}=$}
\end{tabular}
&
\textbf{$\Exp{X}$=} 
&
\textbf{$\Var{X}=$} 
&
\textbf{Beispiele}
\\\addlinespace[0.25em]\midrule\addlinespace[0.25em]
\begin{tabular}{c}
Gleichverteilung\\
$X\sim\cU(a,b)$
\end{tabular} &
Zufällige Wahl eines Punktes in $[a,b]$. 
&
$[a,b]$ &
$\begin{cases}
\frac{1}{b-a}, & \text{für }a\leq t\leq b,\\
0,&\text{sonst.}
\end{cases}
$
&
$\begin{cases}
0, & \text{für }t<a,\\
\frac{t-a}{b-a}, &\text{für }a\leq t\leq b,\\
1,&\text{für }t>b.
\end{cases}$ &
$\displaystyle\frac{a+b}{2}$ &
$\displaystyle\frac{1}{12}(b-a)^2$
\\\addlinespace[0.25em]\hline\addlinespace[0.25em]
\begin{tabular}{c}
Exponential\\
$X\sim Exp(\lambda)$\\
``gedächtnislos''\\
$\cProb{X>t+s}{X>s}$\\
$=\Prob{X>t}$
\end{tabular} &
Analogon der geometrischen Verteilung. Modell für Wartezeiten oder Lebensdauern.
Wartezeit, bis zum eintreffen eines Ereignisses. $\lambda>0$ steht für die
erwartete Anzahl Ereignisse pro Einheitsintervall. &
$[0,\infty)$ &
$\begin{cases}
\lambda e^{-\lambda t}, & \text{für } t\geq 0,\\
0, & \text{für }t<0.
\end{cases}$ &
$\begin{cases}
1-e^{-\lambda t}, & \text{für }t\geq 0,\\
0, & \text{für }t<0.
\end{cases}$ &
$\displaystyle\frac{1}{\lambda}$ (MTBF) &
$\displaystyle\frac{1}{\lambda^2}$ 
\\\addlinespace[0.25em]\hline\addlinespace[0.25em]
\begin{tabular}{c}
Standard-Normal\\
$X\sim\cN(0,1)$
\end{tabular} &
Glockenkurve mit Erwartungswert $\mu=0$ und Varianz $\sigma^2=1$.
&
$\R$ 
&
$\varphi(t)=\frac{1}{\sqrt{2\pi}}e^{-\frac{1}{2}t^2}$ 
&
$\Phi(t)=\int_{-\infty}^{t}\varphi(s) \ ds$ 
&
$0$ 
&
$1$
\\\addlinespace[0.25em]\hline\addlinespace[0.25em]
\begin{tabular}{c}
Normalverteilung, NV,\\
oder Gauss-Verteilung mit\\
Parameter $\mu\in\R$, $\sigma^2 > 0$\\
$X\sim\cN(\mu,\sigma^2)$
\end{tabular} &
\begin{tabular}{c}
Ist $Y\sim\cN(\mu,\sigma^2)$, so ist\\\addlinespace[0.5em]
$X:=\frac{Y-\mu}{\sigma}\sim \cN(0,1)$
\end{tabular}
&
$\R$ 
&
$\frac{1}{\sigma\sqrt{2\pi}}e^{-\frac{(t-\mu)^2}{2\sigma^2}}
=\frac{1}{\sigma}\varphi(\frac{t-\mu}{\sigma})$ 
&
$\Prob{\frac{X-\mu}{\sigma}\leq\frac{t-\mu}{\sigma}}=\Phi(\frac{t-\mu}{\sigma})$
& 
$\mu$ 
& 
$\sigma^2$ 
&
Streuung von Messwerten um ihren Mittelwert, Gewichte oder Grössen von
Individuen einer grössen Bevölkerung, Leistungen in IQ-Tests, Renditen von
Aktien, usw.
\\\addlinespace[0.25em]\hline\addlinespace[0.25em]
\begin{tabular}{c}
Cauchy\\
$X\sim Cauchy(s,t)$
\end{tabular}
&
Breitenparameter $s>0$ und 
Lageparameter $t$, $\abs{t}<\infty$
&
$\R$ &
$\displaystyle \frac{1}{\pi} \cdot \frac{s}{s^2 + (x-t)^2}$ &
$\displaystyle \frac{1}{2}+\frac{1}{\pi}\arctan\left(\frac{x-t}{s}\right)$ &
\begin{tabular}{c}
existiert nicht\\
(nicht abs. integ.)
\end{tabular} &
existiert nicht
\\\addlinespace[0.25em]\hline\addlinespace[0.25em]
\begin{tabular}{c}
Pareto\\
$X\sim Par(\alpha,x_0)$
\end{tabular}
&
Whs., dass Schadengrösse $X$ grösser als $x_0>0$ ist für Param. $\alpha>0$
(``Excess-of-Loss'').
$X\sim\cU(0,1)\Longrightarrow \frac{1}{X}\sim Par(1,1)$.
&
$[x_0,\infty)$ &
$\begin{cases}
  \frac{\alpha x_0^\alpha}{t^{\alpha+1}} & t\geq x_0 \\
  0 & t<x_0
\end{cases}$ &
$1-(\frac{x_0}{t})^{\alpha}$ &
$\begin{cases}
  x_0 \frac{\alpha}{\alpha-1} & \alpha > 1,\\
  \infty                      & \alpha \leq 1.
\end{cases}$ &
$\begin{cases}
  x_0^2 \frac{\alpha}{(\alpha-2)(\alpha-1)^2} & \alpha > 2, \\
 \infty & 1 < \alpha \leq 2.
\end{cases}$
\\\addlinespace[0.25em]\hline\addlinespace[0.25em]
\begin{tabular}{c}
Chi-Quadrat-Verteilung\\
mit $n$ Freiheitsgraden\\
$X\sim\chi_n^2$
\end{tabular} 
&
\begin{tabular}{p{\beschrXstetL}}
Sind $Y_1,\ldots,Y_n\riid\cN(0,1)$, so ist\\ 
\multicolumn{1}{c}{$\displaystyle X:=\sum_{i=1}^n Y_i^2\sim\chi_n^2.$}
\end{tabular}
&
$[0,\infty)$ 
&
\multicolumn{2}{l}{
\boxed{
\begin{tabular}{p{9cm}}
Für $n=2$ ergibt das $Exp(\frac{1}{2})$.
Spezialfall einer $Ga(\alpha,\lambda)$-Verteilung mit
$\alpha=\frac{1}{2}$ und $\lambda=\frac{1}{2}$.
\end{tabular}
} 
}
%$\displaystyle\frac{1}{
% 2^{\frac{n}{2}}\Gamma(\frac{n}{2})}y^{\frac{n}{2}-1}e^{-\frac{1}{2}y}$ 
%& 
%zu kompliziert 
&
$n$ 
&
$2n$
\\\addlinespace[0.25em]\hline\addlinespace[0.25em]
\begin{tabular}{c}
Student-$t$-Verteilung\\
mit $n$ Freiheitsgraden\\
$X\sim t_n$
\end{tabular} &
\begin{tabular}{p{\beschrXstetL}}
Sind $Y$ und $Z$ unabhängig mit $Y\sim\cN(0,1)$ und $Z\sim\chi^2_n$, so ist
\\
\multicolumn{1}{c}{
$\displaystyle X:=\frac{Y}{\sqrt{\frac{1}{n}Z}} \sim t_n.$
}
\end{tabular}
&
$\R$ 
&
\multicolumn{2}{l}{
\boxed{
\begin{tabular}{p{9cm}}
Für $n=1$ ist das eine Cauchy-Verteilung, und für $n\to\infty$ erhält man eine
$\cN(0,1)$-Verteilung. Ist wie die Normalverteilung glockenförmig und
symmetrisch um $0$, aber langschwänziger, und zwar umso mehr, je kleiner $n$
ist.
\end{tabular}}
}
%$\frac{\Gamma(\frac{n+1}{2})}{\sqrt{n\pi}\Gamma(\frac{n}{2})}
%\left(1+\frac{t^2}{n}\right)^{-\frac{n+1}{2}}$ 
%&
%zu kompliziert
&
$0$ für $n>1$
&
$\displaystyle \frac{n}{n-2}$ für $n>2$
\\\addlinespace[0.25em]\bottomrule\addlinespace[0.25em]
\end{tabular}
}

\end{landscape}

\newpage

\section{Standard-Tests \small (für (approximativ) normalverteilte ZV)}

\sep

\subsection{Einstichproben-Tests}

\begin{tabular}{lcc}
\toprule\addlinespace[0.5em]
&
\multicolumn{1}{c}{\textbf{$z$-Test}}
&
\multicolumn{1}{c}{\textbf{$t$-Test}}
\\\addlinespace[0.25em]
\textbf{Beschrieb:} 
& 
\begin{tabular}{p{7cm}}
Normalverteilung, Test für Erwartungswert bei \emph{bekannter} Varianz
\end{tabular}
&
\begin{tabular}{p{7cm}}
Normalverteilung, Test für Erwartungswert bei \emph{unbekannter} Varianz
\end{tabular}
\\\addlinespace[0.5em]\toprule\addlinespace[0.5em]
\textbf{Stichprobe:}
&
$X_1,\ldots,X_n\riid\cN(\vartheta,\sigma^2)$ unter $P_\vartheta$
&
$X_1,\ldots,X_n\riid\cN(\mu,\sigma^2)$ unter $P_{\vec{\vartheta}}$
\\\addlinespace[0.5em]\hline\addlinespace[0.5em]
\textbf{Bekannt:} 
& 
$\sigma^2$ (Varianz)
&
(nichts)
\\\addlinespace[0.5em]\hline\addlinespace[0.5em]
\textbf{Unbekannt $\vartheta$:}
& 
$\vartheta=\mu$
&
$\vec{\vartheta}=(\mu,\sigma^2)$
\\\addlinespace[0.5em]\hline\addlinespace[0.5em]
\textbf{Hypothese $H_0$:}
&
\begin{tabular}{l}
$\vartheta = \vartheta_0$ (einfach)
\\\addlinespace[0.5em]
$\Theta_0=\set{\vartheta_0}$
\end{tabular}
&
\begin{tabular}{l}
$\mu=\mu_0$ (zusammengesetzt)
\\\addlinespace[0.5em]
$\Theta_0=\set{\mu_0}\times (0,\infty)$\\
\quad\ \ $=\dset{\vec{\vartheta}=(\mu,\sigma^2)}{\mu=\mu_0}$
\end{tabular}
\\\addlinespace[0.5em]\hline\addlinespace[0.5em]
\textbf{Alternative $H_A$:}
&
\begin{tabular}{ccc}
1) $\vartheta>\vartheta_0$
&
2) $\vartheta<\vartheta_0$
&
3) $\vartheta\neq\vartheta_0$
\end{tabular}
&
\begin{tabular}{ccc}
1) $\mu>\mu_0$
&
2) $\mu<\mu$
&
3) $\mu\neq\mu_0$
\end{tabular}
\\\addlinespace[0.5em]\hline\addlinespace[0.5em]
\textbf{Teststatistik $T$:}
&
$\displaystyle
T=\frac{\overline{X}_n-\vartheta_0}{\sigma/\sqrt{n}}\sim\cN(0,1)$ unter
$P_{\vartheta_0}$
&
\begin{tabular}{p{5cm}}
\multicolumn{1}{c}{
$\displaystyle T=\frac{\overline{X}_n-\mu_0}{S/\sqrt{n}}\sim 
t_{n-1}$ unter $P_{\vec{\vartheta}_0}$}
\\\addlinespace[0.5em]
\multicolumn{1}{c}{
$\displaystyle
S^2=\frac{1}{n-1}\sum_{i=1}^{n} (X_i-\overline{X}_n)^2$
}
\end{tabular}
\\\addlinespace[0.5em]\hline\addlinespace[0.5em]
\textbf{Krit. Bereich $K$:}
&
\multicolumn{2}{c}{1) $[c_>,\infty)$
\qquad\qquad
2) $(-\infty, c_<]$
\qquad\qquad
3) $(-\infty,-c_{\neq}]\cup [c_{\neq},\infty)$}
\\\addlinespace[0.5em]\hline\addlinespace[0.5em]
\textbf{Krit. Wert $c$:}
&
\multicolumn{1}{l}{
\begin{tabular}{l}
1) $c_>=\Phi^{-1}(1-\alpha)=z_{1-\alpha}$
\\\addlinespace[0.5em]
2) $c_<=-\Phi^{-1}(1-\alpha)=-z_{1-\alpha}$
\\\addlinespace[0.5em]
3) $c_{\neq}=\Phi^{-1}(1-\frac{\alpha}{2})=z_{1-\frac{\alpha}{2}}$
\\\addlinespace[0.5em]
($z_{1-\alpha}=(1-\alpha)$-Quantil der $\cN(0,1)$-Vert.)
\end{tabular}}
&
\multicolumn{1}{l}{
\begin{tabular}{l}
1) $c_>=t_{n-1,1-\alpha}$
\\\addlinespace[0.5em]
2) $c_<=t_{n-1,\alpha}=-t_{n-1,1-\alpha}$
\\\addlinespace[0.5em]
3) $c_{\neq}=t_{n-1,1-\frac{\alpha}{2}}$   
\\\addlinespace[0.5em]
($t_{m,\gamma}=\gamma$-Quantil der $t_{m}$-Vert.)
\end{tabular}}
\\\addlinespace[0.5em]\hline\addlinespace[0.5em]
\begin{tabular}{l}
\textbf{Verwerfe $H_0$}
\\\addlinespace[0.25em]
\textbf{falls:}
\end{tabular}
&
\multicolumn{1}{l}{
\begin{tabular}{lll}
1) &$\displaystyle\overline{X}_n > c_>\frac{\sigma}{\sqrt{n}} +
\vartheta_0$, & $T>z_{1-\alpha}$
\\\addlinespace[0.5em]
2) &$\displaystyle\overline{X}_n < c_<\frac{\sigma}{\sqrt{n}} +
\vartheta_0$, & $T<-z_{1-\alpha}$
\\\addlinespace[0.5em]
3) &$\displaystyle 
\abs{\frac{\overline{X}_n-\vartheta_0}{\sigma/\sqrt{n}}}>c_{\neq}$, 
& $\abs{T}>z_{1-\frac{\alpha}{2}}$
\end{tabular}}
&
\multicolumn{1}{l}{
\begin{tabular}{lll}
1) & $\displaystyle\overline{X}_n > c_>\frac{S}{\sqrt{n}} +
\vartheta_0$, & $T>t_{n-1}{1-\alpha}$
\\\addlinespace[0.5em]
2) & $\displaystyle\overline{X}_n < c_<\frac{S}{\sqrt{n}} +
\vartheta_0$, & $T<-t_{n-1}{1-\alpha}$
\\\addlinespace[0.5em]
3) & $\displaystyle 
\abs{\frac{\overline{X}_n-\vartheta_0}{S/\sqrt{n}}}>c_{\neq}$,
& $\abs{T}>t_{n-1,1-\frac{\alpha}{2}}$
\end{tabular}}
\\\addlinespace[0.5em]\bottomrule
\end{tabular}
\begin{multicols*}{2}
\raggedcolumns

$\text{unabh.} 
\  \stackrel{\neq}{\Longrightarrow} \ 
\text{paarw. unabh.}
\  \Longrightarrow \ 
\text{paarw. unkorr.}
$

Unabhängigkeit von ZV bleibt bei Transf. erhalten!

\sep

$
Y=g(X)\Longrightarrow \Exp{Y}=\Exp{g(X)}=\int_{-\infty}^\infty g(x)f_X(x)\ dx
$
(Absolute Konvergenz nötig)

$
Y=\sum_{\ell=1}^m X_\ell\Longrightarrow
\Exp{Y} = a + \sum_{\ell=1}^{m}b_\ell\Exp{X_\ell}.
$

$\forall\omega\in\Omega: X(\omega)\leq Y(\omega)\Longrightarrow \Exp{X}\leq
\Exp{Y}$

$\cW(X)=\N_0\Rightarrow \Exp{X} = \sum_{j=1}^{\infty} \Prob{X\geq j} = \sum_{\ell=0}^{\infty} P[X >
\ell].$

$
X_i,\ldots,X_n\text{ unabh. }\Rightarrow
\Exp{\prod_{i=1}^n X_i} = \prod_{i=1}^n \Exp{X_i}
$

\sep

$\Exp{\Exp{X}}=\Exp{X}$\\
$\Exp{\Exp{X}\Exp{X}}=(\Exp{X})^2$\\
$\Exp{\Exp{X} X}=\Exp{X}\Exp{X}=(\Exp{X})^2$\\
\Com EW ist innerhalb von EW eine Konstante.

\sep

$\Exp{X^2}<\infty\Rightarrow\Var{X}:=\Exp{(X-\Exp{X})^2}$ (definiert)

(mit $g(X)=(X-\Exp{X})^2$, mittlere q. Abw. von EW)

$\Exp{X^2}<\infty\Rightarrow\Rightarrow \Var{X}=\Exp{X^2}-(\Exp{X})^2$

$\Exp{X^2}<\infty\Rightarrow\Rightarrow\Var{aX+b}=a^2\Var{X}$

$
\Var{\sum_{i=1}^n X_i} = 
\sum_{i=1}^n \Var{X_i} + 2\sum_{i<j} \Cov{X_i,X_j}
$\\
$\Var{X_1+X_2} = \Var{X_1} + \Var{X_2} + 2\Cov{X_1,X_2}.$\\
$
X_i,X_j\text{ p.w. unkorr.}\Rightarrow
\Var{\sum_{i=1}^n X_i} = \sum_{i=1}^n\Var{X_i}
$

\sep

$\Cov{X,Y}=\Exp{XY}-\Exp{X}\Exp{Y}$ oder\\
$\Cov{X,Y}=\Exp{(X_1-\Exp{X_1})(X_2-\Exp{X_2})}$\\
$X,Y \text{ unabh.}\Longrightarrow \Cov{X,Y}=0$\\
$X,Y \text{ unabh.}\Longrightarrow \Exp{XY}=\Exp{X}\Exp{Y}$\\
$\Cov{X,Y}=\Cov{Y,X}$\\
$\Cov{X,X}=\Var{X}$,\\
$\Cov{X,Y+Z}=\Cov{X,Y}+\Cov{X,Z}$\\
$\Cov{X,a}=0 \text{ für }a\in\R$\\
$\Cov{c+aX,d+bY}=ab\Cov{X,Y}$

\end{multicols*}

\subsection{Zweistichproben-Tests}

\begin{tabular}{Hcc}
\toprule\addlinespace[0.5em]
&
\textbf{Gepaart $(m=n)$}
&
\textbf{Ungepaart $(m\neq n\lor m=n)$}
\\\addlinespace[0.25em]
\textbf{Beschrieb} 
&
\begin{tabular}{p{7cm}}
Normalverteilung, Test für Erwartungswert bei unbekannter Varianz
\end{tabular}
&
\begin{tabular}{p{7cm}}
Normalverteilung, Test für Erwartungswert bei unbekannter Varianz
\end{tabular}
\\\addlinespace[0.5em]\toprule\addlinespace[0.5em]
\textbf{Stichprobe}
&
\multicolumn{2}{c}{
\begin{tabular}{c}
$X_1,\ldots,X_n\riid\cN(\mu_X,\sigma^2)$ unter $P_\vartheta$ (oder
$P_{\vec{\vartheta}}$) \\\addlinespace[0.5em]
$Y_1,\ldots,Y_m\riid\cN(\mu_Y,\sigma^2)$ unter $P_\vartheta$ (oder
$P_{\vec{\vartheta}}$)
\\\addlinespace[0.5em]
$X_1,\ldots,X_n$, $Y_1,\ldots,Y_m$ sind unabh. unter $P_\vartheta$ (oder
$P_{\vec{\vartheta}}$)
\\\addlinespace[0.5em]
Varianz $\sigma^2$ ist in beiden Stichproben dieselbe
\end{tabular}
}
\\\addlinespace[0.5em]\hline\addlinespace[0.5em]
\textbf{Vorgehen}
&
\begin{tabular}{p{9.25cm}}
Man bildet paarweise Differenzen:
\\\addlinespace[0.5em]
\multicolumn{1}{c}{
$\displaystyle Z_i:=X_i-Y_i\riid\cN(\mu_X-\mu_Y,2\sigma^2)$
}
\\\addlinespace[0.5em]
\begin{enumerate}[label=\alph*)]
  \item \emph{$\sigma^2$ bekannt}: \textbf{$z$-Test}
  \[
  T =
  \frac{\overbrace{(\overline{X}_n-\overline{Y}_n)}^{\overline{Z}_n}
  -\overbrace{(\mu_X-\mu_Y)}^{\mu_{\vartheta_0}}}{
  \sigma\sqrt{\frac{2}{n}}
  }\sim\cN(0,1)
  \]
  unter jedem $P_\vartheta$.
  \item \emph{$\sigma^2$ unbekannt}: \textbf{$t$-Test}
  \[
  T =
  \frac{(\overline{X}_n-\overline{Y}_n)
  -(\mu_X-\mu_Y)}{
  S\sqrt{\frac{2}{n}}
  }\sim t_{\color{red}{n}}
  \]
  unter jedem $P_{\vec{\vartheta}}$, wobei\newline
  \[
  \color{red}{
  S^2 = \frac{1}{2}(S_X^2 + S_Y^2) = \frac{1}{2}S_Z^2.
  }
  \]
  $S_X^2=\frac{1}{n-1}\sum_{i=1}^n (X_i-\overline{X}_n)^2$ \quad ($S_Y$, $S_Z$
  analog)
\end{enumerate}
\end{tabular}
&
\begin{tabular}{p{9.25cm}}
Man kann nicht mehr paarweise Differenzen bilden, auch wenn zufällig $m=n$ ist.
\\
\begin{enumerate}[label=\alph*)]
  \item $\sigma^2$ \emph{bekannt}: \textbf{$z$-Test}
\[
T=\frac{(\overline{X}_n-\overline{Y}_m)-(\mu_X-\mu_Y)}{
\sigma\sqrt{\frac{1}{n}+\frac{1}{m}}}
\sim \cN(0,1)
\]
unter jedem $P_\vartheta$.
\item \emph{$\sigma^2$ unbekannt}: \textbf{$t$-Test}
\[
T=\frac{(\overline{X}_n - \overline{Y}_m)-(\mu_X-\mu_Y)}{
S\sqrt{\frac{1}{n}+\frac{1}{m}}
}\sim t_{n+m-2}
\]
unter jedem $P_{\vec{\vartheta}}$, wobei ($S$=Schätzer für $\sigma^2$)\newline
\begin{tabular}{l}
\addlinespace[0.5em]
$S_X^2=\frac{1}{n-1}\sum_{i=1}^n (X_i-\overline{X}_n)^2$
\\\addlinespace[0.5em]
$S_Y^2=\frac{1}{m-1}\sum_{i=1}^m (Y_i-\overline{Y}_m)^2$
\\\addlinespace[1em]
$
S^2=\frac{1}{m+n-2}\left((n-1)S_X^2 + (m-1)S_Y^2\right).
$
\end{tabular}
\end{enumerate}
\end{tabular}
\\\bottomrule
\end{tabular}
\begin{multicols*}{2}

\raggedcolumns

Hier sieht man auch die vielen Möglichkeiten $f_X(x)$ zu berechnen:
\begin{align*}
f_X(x) &= \frac{d}{dx} F_X(x) = \frac{d}{dx}\lim_{y\to\infty} F(x,y)
\\
&=\frac{d}{dx}\lim_{y\to\infty}\overbrace{\left(\int_{-\infty}^x\int_{-\infty}^y
f(u,v) \ dv \ du\right)}^{=F(x,y)} 
\\
&=\frac{d}{dx}\left(\int_{-\infty}^x\int_{-\infty}^\infty f(u,v)
\ dv \ du\right)
=\int_{-\infty}^\infty f(x,v) \ dv.
\end{align*}

\sep

\Thm[Additionsregel]
  \[
  P[A\cup B] = P[A] + P[B] - P[A\cap B]
  \]

\sep

\Lem Für beliebige $A\subseteq\Omega$ ist dann in einem Laplace-Raum:
\begin{gather*}
P[A]=\frac{\text{Anz. Elementarereignisse in }A}{\text{Anz. Elementarereignisse
in
}\Omega}=\frac{\card{A}}{\card{\Omega}}
=\frac{\text{``günstige''}}{\text{``mögliche''}}
\end{gather*}

\Com Abzählen: Kombinatorik

\sep

\Thm[Formel von Bayes] Ist $A_1,\ldots,A_n$ eine Zerlegung von $\Omega$ mit
$P[A_i]>0$ für $i=1,\ldots,n$ und $B$ ein Ereignis mit $P[B]>0$, so gilt für
jedes $k$
\[
\cProb{A_k}{B} \stackrel{\text{Def}}{=}
\frac{\Prob{A_k\cap B}}{\Prob{B}} 
=
\frac{\overbrace{\cProb{B}{A_k}\Prob{A_k}}^{\mathclap{=\Prob{A_k\cap
B}\text{
(Mult. Regel)}}}}{\underbrace{\sum_{i=1}^{n}
\cProb{B}{A_i}\Prob{A_i}}_{\mathclap{=\Prob{B}\text{
(Satz. d. tot. Whs.)}}}}.
\]

\sep

\Def[(Stoch.) Unabh. \emph{Ereignisse}]
\begin{gather*}
P[A\cap B] = P[A]P[B]. \quad\text{bzw.}\quad P\left[\bigcap_{i=1}^{m}
A_{k_i}\right] =
\prod_{i=1}^{m} P[A_{k_i}].
\end{gather*}
Ist $P[A]=0$ oder $P[B]=0$, so sind $A$ und $B$ immer unabhängig. Für $P[A]\neq
0$ gilt
\[
A,B\text{ unabhängig } \Longleftrightarrow P[B|A] = P[B].
\]

\begin{comment}
\newpage

\part{Leftovers}

\section{Wahrscheinlichkeiten}

\Def[$\sigma$-Algebra] ist ein Mengensystem $\cF\subseteq \cP(\Omega)$,
wenn folgende Punkte erfüllt sind
\begin{enumerate}[label=\roman*)]
  \item $\Omega\in\cF$,
  \item für jedes $A \in \cF$ gilt: $A^c\in\cF$,
  \item für jede Folge $(A_n)_{n\in\N}$ mit $A_n\in\cF$ für alle $n\in\N$ ist
  auch die Vereinigung $\bigcup_{n=1}^\infty A_n\in \cF$.
\end{enumerate}

\sep

\Def[Wahrscheinlichkeitsmass] ist eine Abbildung $P\colon\cF\to [0,1]$,
welche die nachfolgenden Axiome erfüllt. Für $A\in\cF$ nennen wir $P[A]\in[0,1]$
die \emph{Wahrscheinlichkeit} (kurs WS), dass $A$ eintritt. Die geforderten
Grundregeln (Axiome) sind:

\begin{enumerate}[label=A\arabic*),leftmargin=22pt]
  \item $P[A]\geq 0$ für alle Ereignisse $A\in\cF$.
  \item $P[\Omega]=1.$
  \item $P\left[\bigcup_{i=1}^\infty A_i\right]=\sum_{i=1}^\infty
  P[A_i]$, sofern die $A_i\in\cF$ \emph{paarweise disjunkt} sind, d.h. falls gilt
  $A_i\cap A_j = \emptyset$ für $i\neq j$. \
  \txtcom{genannt: $\sigma$-Additivität}
\end{enumerate}

\Com Auch $A1)$ und $A2)$ benutzen, dass $\cF$ eine $\sigma$-Algebra ist, weil
ja die Argumente von $P$ Mengen aus $\cF$ sein müssen.

\section{Diskrete Zufallsvariablen und Verteilungen}

\subsection{Grundbegriffe}

\Def[diskrete Zufallsvariable] Funktion
$X\colon\Omega\to\R$. Mit $\Omega$ ist natürlich auch der Wertebereich
$\cW(X)=\{x_1,x_2,\ldots\}$ von $X$ endlich oder abzählbar.

\Com Zufallsvariable $X$ partitioniert $\Omega$ in Äquivalenzklassen. Alle
$x_k\in\cW(X)$ sind repräsentanten für eine Äquivalenzklassen in $\Omega$ (dem
Urbild von $x_k$, das für alle $x_k$ verschieden sein muss, weil $X$ eine
Funktion ist.) \todo{wirklich verschieden?}

\sep
 
\Cor[Gedächtnislosigkeit von Exponentialvert.]

\begin{gather*}
\begin{align*}
\cProb{X>t+s}{X>s}
&=\frac{\Prob{X>t+s,X>s}}{\Prob{X>s}}
=\frac{\Prob{X>t+s}}{\Prob{X>s}}
\\
&=\frac{1-\Prob{X\leq t+s}}{1-\Prob{X\leq s}}
=\frac{1-F_X(t+s)}{1-F_X(s)}
\\
&=\frac{1-(1-e^{-\lambda(t+s)})}{1-(1-e^{\lambda s})}
=\frac{e^{-\lambda(t+s)}}{e^{\lambda s}}=e^{-\lambda t}
\\
&=1-F_X(t)=\Prob{X>t}.
\end{align*}
\end{gather*}

\sep

\todo{Kurzbeispiel: Monte-Carlo-Integration (siehe ZF Fabian Hahn)}
\end{comment}

\end{multicols*}
\end{document}
